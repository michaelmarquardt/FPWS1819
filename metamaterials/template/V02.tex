\documentclass[final,hyperref={pdfpagelabels=false}]{beamer}
\usepackage[ngerman]{babel}
\usepackage[utf8]{inputenc}
\usepackage{xcolor}
\selectcolormodel{cmyk}
\usepackage[orientation=portrait,size=a0, debug]{beamerposter}
\usepackage{helvet}
\usepackage{chemformula}
\usepackage[style=numeric-comp,backend=biber,sortlocale=de_DE,natbib=true]{biblatex}
\addbibresource{Quellen.bib}
\usepackage{pgfplots}
\usepackage[separate-uncertainty=true,
            list-units = single,
            list-separator = {;\,},
            list-final-separator = {\,und\,},%
            list-pair-separator = {\,und\,},%
            exponent-product=\cdot,%
            output-decimal-marker={,},%
            range-phrase={\,bis\,}%
            ]{siunitx}
\usepackage{mathtools}
\usepackage{physics}
\usetikzlibrary{
    arrows,
    calc,
    %decorations.markings,
    math,
    %intersections,
    %shapes.arrows,
    arrows.meta,
    patterns
    }
%\tikzset{
%    my arrow/.style={
%        fill=black,
%        inner sep=3pt,
%        shape=single arrow,
%        minimum width=0.5cm,
%        minimum height=1.5cm,
%        outer sep=0pt,
%        shape border rotate=90
%       
%    }
\usepackage{varwidth}
\newcommand\Umbruch[2][0.15\linewidth]{
    \begin{varwidth}{#1}
        \flushleft #2   
    \end{varwidth}}
\newenvironment{indentitemize}{%
    \begin{itemize}%
    \addtolength{\itemindent}{2ex}%
    \setlength{\leftmargin}{2ex}%
    \setlength{\topsep}{0pt}%
    }
    {%
\end{itemize}%
}
\newcommand*\circled[1]{\tikz[baseline=(char.base)]{
            \node[shape=circle,draw,inner sep=.4ex, line width = 1.5mm] (char) {#1};}}

\usepackage{adjustbox} % For minipage alignment

%\usepackage{array}
%\newcolumntype{L}[1]{>{\raggedright\let\newline\\\arraybackslash\hspace{0pt}}m{#1}}
%\newcolumntype{C}[1]{>{\centering\let\newline\\\arraybackslash\hspace{0pt}}m{#1}}
%\newcolumntype{R}[1]{>{\raggedleft\let\newline\\\arraybackslash\hspace{0pt}}m{#1}}

%%%%%%%%%%%%%%%%%%%%%%%%%%%%%%%%%%%%%%%%%%%%%%%%%%%%%%%%%%%%%%%%%%%%%%%%%%%%%%%%%5
\graphicspath{{figures/}}
\title[Raman-Spektroskopie]{Raman Effekt}
\author[Meiler \& Marquardt]{Tim Meiler \& Michael Marquardt}
\institute[Universität Stuttgart]{Universität Stuttgart}
\date{18.06.2018}

%%%%%%%%%%%%%%%%%%%%%%%%%%%%%%%%%%%%%%%%%%%%%%%%%%%%%%%%%%%%%%%%%%%%%%%%%%%%%%%%%5
\mode<presentation>
{
\usetheme{Dreuw}
}
\begin{document}
\begin{frame}{}
\begin{columns}[t]
    \begin{column}{.48\linewidth}
        \begin{block}{\circled{1} Versuchsziel und Versuchsmethode}
        \textbf{Methode:} Aufnahme der Schwingungsspektren von Molek"ulen mittels Raman Effekt\\
        \textbf{Proben:} Tetraedermoleküle \ch{CH_2Cl_2}, \ch{CHCl_3}, \ch{CDCl_3}, \ch{CHBr_3} und \ch{CDBr_3} (Zuordnung unbekannt)
        \begin{indentitemize}
            \item Bestimmung der Boltzmann-Konstanten aus dem Stokes-Antistokes-Intensit"atsverh"altnis
            \item Molek"ule den Proben zuordnen durch Vergleich der Positionen der Raman-Linien
            %Zuordnen der Proben zu den Tetraedermolek"ulen durch Positionen der Raman-Linien
            \item Depolarisationsgrade aus Intensitätsverhältnis paralleler und senkrechter Polarisation
        \end{indentitemize}
        \end{block}
    \end{column}
    \hfill
    \begin{column}{.48\linewidth}
        \begin{block}{\circled{9} Literatur}
            \printbibliography
        \end{block}
    \end{column}
\end{columns}
\end{frame}
\end{document}