\captionsetup{
    singlelinecheck=false,
    justification=justified
    }
\begin{figure}[htbp]
\begin{subfigure}[t][][t]{0.48\textwidth}
    \begin{tikzpicture}
    [x=0.1\textwidth, y=0.1\textwidth]
    \draw (0,0) rectangle (10,5);
    \draw[thick] (0.5,4.5) circle(0.3) node{1};
    \filldraw[orange]
        (1,2.2) .. controls (0.8,2.5) ..
        (1.2,3.5) .. controls (1.6,2.5) ..
        (1.4,2.2);
    \filldraw[red]
        (1.2,3) .. controls (1.4,2.5) ..
        (1.2,2.3) .. controls (1,2.5) ..
        (1.2,3);
    \draw[fill=darkgrey]
        (0.2,0.2)--
        (2.2,0.2)--
        (1.4,1.2)--
        (1.4,2.2)--
        (1,2.2)--
        (1,1.2)--
        (0.2,0.2);
    \draw[fill=brightgrey]
        (1.2,3.5)--
        (0.4,3.2)--
        (0.38,3.25)--
        (1.2,3.6)--
        (1.6,3.63)--
        (1.8,3.7)--
        (5,3.7)--
        (5,3.4)--
        (1.8,3.4)--
        (1.6,3.47)--
        (1.2,3.5);
    \filldraw[red,scale=0.5,shift={(11.5,7)}]
        (0,.5)--
        (2,.5)--
        (2,1)--
        (3,0)--
        (2,-1)--
        (2,-.5)--
        (0,-.5);
%    \draw[fill=red]
%        (4,1)--
%        (6,1)--
%        (6,0.5)--
%        (7,1.5)--
%        (6,2.5)--
%        (6,2)--
%        (4,2)--
%        (4,1);
    \draw[fill=brightgrey,shift={(8.5,4.5)}]
        (0,0)--
        (0.025,0.05)--
        (0.525,-0.2)..controls(0.6,-0.25)..
        (0.525,-0.3)--
        (-0.175,-0.65)--
        (-0.16,-1)--
        (-0.11,-1.1)--
        (-0.11,-4)--
        (-0.365,-4)--
        (-0.365,-1.1)--
        (-0.315,-1)--
        (-0.3,-0.65)..controls(-0.25,-0.6)..
        (-0.2,-0.57)--
        (0.48,-0.25)--
        (0,0);
     % text
     \draw[] (1.5,0.5)--(2.5,0.5) node[right]{Bunsen burner};
     \draw[] (2.5,3.5)--(2.5,2.9) node[right]{glass pipette};
     \draw[] (8.45,4.5)--(8,4.5) node[left]{reduce the size of the opening};
    \end{tikzpicture}
    \subcaption{
        %Bending of a glass pipette through melting.
        A glass pipette is formed through melting.
        The new form ensures the right flow behavior in step 2.
        The opening of the pipette is hold into the fire to reduce the openings size.
        This ensures the right flow velocity.
        }
\end{subfigure}
\hfill
\begin{subfigure}[t][][t]{0.48\textwidth}
    \begin{tikzpicture}
    [x=0.1\textwidth, y=0.1\textwidth]
    \draw (0,0) rectangle (10,5);
    \draw[thick] (0.5,4.5) circle(0.3) node{2};
    \filldraw[water] (0.5,0.5) rectangle (9.5,2.2);
    \draw[fill=grey]
        (0.2,2.5)--
        (0.5,2.5)--
        (0.5,0.5)--
        (9.5,0.5)--
        (9.5,2.5)--
        (9.8,2.5)--
        (9.8,0.2)--
        (0.2,0.2)--
        (0.2,2.5);
    \draw[fill=solution,shift={(5,1.5)},rotate=250,scale=1.2]
        (0,0)--
        (0.025,0.05)--
        (0.525,-0.2)..controls(0.6,-0.25)..
        (0.525,-0.3)--
        (-0.175,-0.65)--
        (-0.16,-1)--
        (-0.11,-1.1)--
        (-0.11,-4)--
        (-0.365,-4)--
        (-0.365,-1.1)--
        (-0.315,-1)--
        (-0.3,-0.65)..controls(-0.25,-0.6)..
        (-0.2,-0.57)--
        (0.48,-0.25)--
        (0,0);
    \draw[fill=brightgrey,shift={(5,1.5)},rotate=250,scale=1.2]
        (-0.11,-4)--
        (-0.11,-2)--
        (-0.365,-1.3)--
        (-0.365,-4)--
        (-0.11,-4);
    \draw[line width=0.006\textwidth,dotted,color=solution]
        (5.02,1.5) to[thick] (5.02,2.2);
    \foreach \i in {0.55,0.66,...,2.55}
        \filldraw[sphere] (\i,2.25) circle (0.05);
    \foreach \i in {7.45,7.56,...,9.45}
        \filldraw[sphere] (\i,2.25) circle (0.05);
    \draw[sphere,->,line width=0.007\textwidth]
        (5.2,1.5)..controls(5.2,2.2)and(5.35,2.25)..
        (7,2.25);
    % text
    \draw[] (9.65,2.5)--(9.65,4.5) node[left]{Petri dish};
    \draw[] (9.3,2.3)--(9.3,4) node[left]{microsphere monolayer};
    \draw[] (8.95,1.5)--(8.95,3.5) node[left]{distilled water};
    \draw[] (2.2,3.5) node[right]{particle solution};
    \draw[] (3.8,2.2)--(3.8,3.3);
    \end{tikzpicture}
    \subcaption{
        %Creation of a microsphere monolayer at the water surface.
        The pipette is filled with the particle solution and hold into a petri dish with distilled water.
        The solvent rises and takes the microspheres with it.
        The microspheres accumulate in a hexagonal monolayer at the border of the petri dish.
        }
\end{subfigure}

\begin{subfigure}[t][][t]{0.48\textwidth}
    \begin{tikzpicture}
    [x=0.1\textwidth, y=0.1\textwidth]
    \draw (0,0) rectangle (10,5);
    \draw[thick] (0.5,4.5) circle(0.3) node{3};
    \filldraw[water] (0.5,0.5) rectangle (9.5,1.44);
    \draw[fill=grey]
        (0.2,2.5)--
        (0.5,2.5)--
        (0.5,0.5)--
        (9.5,0.5)--
        (9.5,2.5)--
        (9.8,2.5)--
        (9.8,0.2)--
        (0.2,0.2)--
        (0.2,2.5);
    \foreach \i in {2.25,2.14,...,1.49}
        \filldraw[sphere] (0.55,\i) circle (0.05);
    \foreach \i in {0.55,0.66,...,5.55}
        \filldraw[sphere] (\i,1.49) circle (0.05);
    \draw[fill=substrate]
        (1,0.5) rectangle (2,.7)
        (3,0.5) rectangle (4,.7);
    \draw[sphere,->,ultra thick]
        (1.5,1.3) -- (1.5,0.8);
    \draw[sphere,->,ultra thick]
        (3.5,1.3) -- (3.5,0.8);
    \draw[ultra thick,fill=water]
        (8,1)..controls(8.5,3.8)..
        (10,4.5)--
        (10,4)..controls(9,3.5)..
        (8.5,1);
    \draw[ultra thick,->]
        (8.35,1.4)--(8.5,2.12);
    % text
    \draw[] (9.6,4)--(9,4.5) node[left]{suck away the water through the hose};
    \draw[] (1.8,0.6)--(1.8,3.5) node[right]{substrate};
    \draw[] (2.5,1.5)--(2.5,2.5) node[right]{monolayer subsides};
    \end{tikzpicture}
    \subcaption{
        %Applying the monolayer to the substrates.
        %A napkin is dunked at one point of the petri dish border.
        %It absorbs water from the ground and causes the microspheres to travel to the other site of the petri dish.
        %This makes the monolayer more dense.
        Substrate plates are placed under the monolayer.
        A hose is used to suck the most of the water away until the monolayer almost touches the substrate.
        The rest of the water evaporates over night.
        }
\end{subfigure}
\hfill
\begin{subfigure}[t][][t]{0.48\textwidth}
    \begin{tikzpicture}
    [x=0.1\textwidth, y=0.1\textwidth]
    \draw (0,0) rectangle (10,5);
    \draw[thick] (0.5,4.5) circle(0.3) node{4};
    \draw[fill=substrate]
        (0.5,0.5) rectangle (2.5,0.7)
        (0.5,1.5) rectangle (2.5,3.5);
    \foreach \i in {1.63,1.9764,...,3.41}
        \foreach \j in {0.6,0.8,...,2.41}
            \filldraw[sphere] (\j,\i) circle (0.095);
    \foreach \i in {1.8032,2.1496,...,3.41}
        \foreach \j in {0.7,0.9,...,2.41}
            \filldraw[sphere] (\j,\i) circle (0.095);
    \foreach \i in {0.6,0.8,...,2.41}
        \filldraw[sphere] (\i,0.795) circle (0.095);
    \coordinate (s) at (7,0);
    \draw[fill=substrate,shift={(s)}]
        (0.5,0.5) rectangle (2.5,0.7)
        (0.5,1.5) rectangle (2.5,3.5);
    \foreach \i in {1.63,1.9764,...,3.41}
        \foreach \j in {0.6,0.8,...,2.41}
            \filldraw[sphere,shift={(s)}] (\j,\i) circle (0.105);
    \foreach \i in {1.8032,2.1496,...,3.41}
        \foreach \j in {0.7,0.9,...,2.41}
            \filldraw[sphere,shift={(s)}] (\j,\i) circle (0.105);
    \foreach \i in {0.6,0.8,...,2.41}
        \filldraw[sphere,shift={(s)}]
            (\i,0.79) ellipse [x radius=0.105,y radius=0.09];
    \filldraw[sphere]
        (4,1.3) ellipse [x radius=.5,y radius=.5]
        (6,1.25) ellipse [x radius=.55,y radius=.45];
    \draw[fill=substrate]
        (3.3,0.5) rectangle (4.7,0.8)
        (5.3,0.5) rectangle (6.7,0.8);
    \filldraw[red,yscale=.8,shift={(3.5,3.5)}]
        (0,.5)--
        (2,.5)--
        (2,1)--
        (3,0)--
        (2,-1)--
        (2,-.5)--
        (0,-.5);
    % text
    \draw[red] (3.5,3.6) node[right]{\SI{108}{\degreeCelsius}};
    \draw[red] (3.5,4.1) node[right]{$\SI{120}{\second}-\SI{240}{\second}$};
    \draw[] (1.5,3.5) node[above]{spheres};
    \draw[] (8.5,3.5) node[above]{ellipsioids};
    \draw[] (2.5,0.8)--(3.5,1.3);
    \draw[] (7.5,0.8)--(6.55,1.25);
    \end{tikzpicture}
    \subcaption{
        %Annealing of the probe shrinks the space between the spheres.
        The mask is annealed in order to deform the microspheres into ellipsoids.
        This closes the gaps between the spheres and leaves only little holes between the spheres.
        }
\end{subfigure}
\end{figure}

\setcounter{figure}{4}
\begin{figure}[htbp]
\begin{minipage}[t][][t]{0.68\textwidth}
    \large
    \begin{color}{mittelblau}
        Preparing the particle solution: \cite{manual}
    \end{color}
    \normalsize
    \begin{enumerate}
        \item Centrifuge \SI{1}{\milli\litre} of a particle solution containing the polystyrene microspheres until sedimentation and remove about $2/3$ of the water.
        \item Loosen the microspheres using an ultrasonic bath (15-\SI{20}{\minute}).
        \item Mix \SI{100}{\micro\litre} of this with \SI{100}{\micro\litre} ethanol and \SI{6}{\micro\litre} hexylamine (\SI{2}{vol}-\si{\percent} in ethanol).
        \item Use the ultrasonic bath again (15-\SI{20}{\minute}).
    \end{enumerate}
\end{minipage}
\hfill
\begin{minipage}[t][][b]{0.28\textwidth}
    \includegraphics[width=\textwidth]{petrischale.jpg}
    \caption{Photography of (2).
        One can see the monolayer as shiny film. \cite{manual}}
\end{minipage}
\end{figure}