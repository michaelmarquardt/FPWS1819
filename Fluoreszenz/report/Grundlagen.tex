\section{Theory}
\subsection{Spontaneous parametric downconversion} 
Spontaneous parametric downconversion is the spontaneous splitting of a photon in two different photons. That can be done by shining with a laser beam on a crystal, which has nonlinear effects. Under certain circumstances the both emitted photons of the crystal are entangled. 
As for every process momentum and energy conservation must be fulfilled. Because the energy for a photon is $E = \hbar \omega$, one can consider the frequencies to satisfiy the energy conservation. The pumping laser from the laser has the frequency $\omega_{\mathrm{p}}$. The resulting photons, which are named as signal and idler photon, after the splitting must have frequencies below $\omega_{\mathrm{p}}$. The energy conservation can be written as
\begin{equation}
    \omega_{\mathrm{p}} = \omega_{\mathrm{i}} + \omega_{\mathrm{s}},
\end{equation}
where $\omega_{\mathrm{i}}$ is the frequency of the idler photon and $\omega_{\mathrm{s}}$ of the signal photon. Here we are interested in the degenerated case $ \omega_{\mathrm{s}} = \omega_{\mathrm{i}}$ to make the photons indistinguishable by frequency. 
As the momentum can be calculated by $\vec{p} = \hbar \vec{k}$, for  the momentum conversation one con consider the $\vec{k}$ vectors of the tree photons. One obtains
\begin{equation}
    \vec{k}_{\mathrm{p}} =  \vec{k}_{\mathrm{i}}  + \vec{k}_{\mathrm{s}},
    \label{eq:phase_matching}
\end{equation}
what is called phase matching. 
The explained process is not possible for free electrons, but in a nonlinear medium. The nonlinearity refers to the polarization which is not linear dependent on the electric field
\begin{equation}
    \vec{P}_{\mathrm{i}} = \varepsilon_0 \left( \chi^{(1)} \vec{E}_{\mathrm{j}} 
    + \chi^{(2)} \vec{E}_1 \vec{E}_2 
    + \chi^{(3)} \vec{E}_1 \vec{E}_2 \vec{E}_3 + ... \right).
\end{equation}
The understand phase matching it is important to understand how photons propagate in a birefringence media. 

%Talk:
%   - both walk off effect
%   - Phase matching
%   - 2 kegel, deren schnittpunkte
%   - ordinary, extraordinary beam
\subsection{CHSH inequality}
%TALK about:
%   - bell states
%   - entanglement
%   - hanbury brown twiss effect
%   - inequality
%   - how to violate inequality
