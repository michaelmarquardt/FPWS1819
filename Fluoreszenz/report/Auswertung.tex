\section{Analysis}
\subsection{Theoretical test of CHSH inequality}
In LHV-theory one assumes that a local hidden variable (LHV) $\lambda$ exists which is created during the creation of a photon pair.
Now each of this photons is measured at a detector after a polarizer with angle $\alpha$/$\beta$.
The corresponding variable is $A(\alpha,\lambda)$/$B(\beta,\lambda)$ which is 1 when the photon is detected and $-1$ otherwise.
The locality of $\lambda$ states that $A(\alpha,\lambda)$ and $B(\beta,\lambda)$ are independent.
One can now define the expectation value
\begin{align}
E(\alpha,\beta)
    &=\int \dd \lambda\ p(\lambda)A(\alpha,\lambda)B(\beta,\lambda)
    \label{eq:a:defE}
\end{align}
for coincidence between $A$ and $B$ with the normalized distribution $\int \dd\lambda\ p(\lambda)=1$.
One can now calculate
\begin{subequations}
\label{eq:a:Eab-Eab'}
\begin{align}
\abs{E(\alpha,\beta)-E(\alpha,\beta')}
    &=\abs{\int \dd \lambda\ p(\lambda)\left[A(\alpha,\lambda)B(\beta,\lambda)-A(\alpha,\lambda)B(\beta',\lambda)\right]}
    \label{eq:a:Eab-Eab':1}\\
    &\leq \int \dd \lambda\ p(\lambda)\abs{A(\alpha,\lambda)}\abs{B(\beta,\lambda)-B(\beta',\lambda)}
    \label{eq:a:Eab-Eab':2}\\
    &\leq \int \dd \lambda\ p(\lambda)\abs{B(\beta,\lambda)-B(\beta',\lambda)}
    \label{eq:a:Eab-Eab':3}
\end{align}
\end{subequations}
and analogously
\begin{align}
\abs{E(\alpha',\beta)+E(\alpha',\beta')}
    &\leq \int \dd \lambda\ p(\lambda)\abs{B(\beta,\lambda)+B(\beta',\lambda)}.
    \label{eq:a:Ea'b-Ea'b'}
\end{align}
Because $B=\pm 1$ it is
\begin{equation}
\abs{B(\beta,\lambda)-B(\beta',\lambda)}+\abs{B(\beta,\lambda)+B(\beta',\lambda)}=2
\label{eq:a:B+B}
\end{equation}
and therefore
\begin{subequations}
\label{eq:a:E+E}
\begin{align}
\abs{E(\alpha,\beta)-E(\alpha,\beta')}+\abs{E(\alpha,\beta)-E(\alpha,\beta')}
    &\leq \int \dd \lambda\ p(\lambda)\left(\abs{B(\beta,\lambda)-B(\beta',\lambda)}+\abs{B(\beta,\lambda)+B(\beta',\lambda)}\right)
    \label{eq:a:E+E:1}\\
    &= \int \dd \lambda\ p(\lambda)2
    \label{eq:a:E+E:2}\\
    &= 2
    \label{eq:a:E+E:3}
\end{align}
\end{subequations}
One can now define the variable S
\begin{equation}
S\equiv E(\alpha,\beta)-E(\alpha,\beta')+E(\alpha',\beta)+E(\alpha',\beta')
\label{eq:a:S}
\end{equation}
which satisfies the CHSH inequality
\begin{subequations}
\label{eq:a:CHSH}
\begin{align}
\abs{S}
    &=\abs{E(\alpha,\beta)-E(\alpha,\beta')+E(\alpha',\beta)+E(\alpha',\beta')}
    \label{eq:a:CHSH:1}\\
    &\leq \abs{E(\alpha,\beta)-E(\alpha,\beta')}+\abs{E(\alpha',\beta)+E(\alpha',\beta')}
    \label{eq:a:CHSH:2}\\
    &\leq 2
    \label{eq:a:CHSH:3}.
\end{align}
\end{subequations}
Let us now take a look at the quantum correlation when no hidden variable exists.
The expectation value $E(\alpha,\beta)$ can be expressed through the coincidence count rates $C(\alpha,\beta)$ for polarizers with angle $\alpha$ and $\beta$ like
\begin{equation}
E(\alpha,\beta)=\frac{C(\alpha,\beta)-C(\alpha,\beta^\perp)-C(\alpha^\perp,\beta)+C(\alpha^\perp,\beta^\perp)}{C(\alpha,\beta)+C(\alpha,\beta^\perp)+C(\alpha^\perp,\beta)+C(\alpha^\perp,\beta^\perp)},
\label{eq:a:EC}
\end{equation}
where $\alpha^\perp=\alpha+\SI{90}{\degree}$.
The coincidence count rates will be measured in the experiments but they can also theoretically be derived by defining the coincidence operator
\begin{align}
\hat A(\alpha,\beta)
    &=\hat J(\alpha)\otimes \hat J(\beta)
    \label{eq:a:A}\\
\hat J
    &=\hat R(\alpha) \begin{pmatrix}
    1&0\\0&0
    \end{pmatrix}\hat R^{-1}(\alpha)
    \label{eq:a:J}\\
\hat R
    &=\begin{pmatrix}
    \cos(\alpha)&\sin(\alpha)\\
    -\sin(\alpha)&\cos(\alpha)
    \end{pmatrix}.
    \label{eq:a:R}
\end{align}
as
\begin{equation}
C(\alpha,\beta)   = \trace{\rho \hat A(\alpha,\beta)}
\label{eq:a:CA}
\end{equation}
with the density matrix $\rho$.
For a set of angles
\begin{subequations}
\label{eq:a:angles}
\begin{align}
\alpha&=\SI{22.5}{\degree}\label{eq:a:angles:a}\\
\alpha'&=\SI{-22.5}{\degree}\label{eq:a:angles:a'}\\
\beta&=\SI{0}{\degree}\label{eq:a:angles:b}\\
\beta'&=\SI{-45}{\degree}\label{eq:a:angles:b'}
\end{align}
\end{subequations}
the S values for the pure bell states become
\begin{subequations}
\label{eq:a:Sbell}
\begin{align}
S(\Phi^{(+)})&=2\sqrt{2}>2\label{eq:a:sbell:phi+}\\
S(\Phi^{(-)})&=0\label{eq:a:sbell:phi-}\\
S(\Psi^{(+)})&=0\label{eq:a:sbell:psi+}\\
S(\Psi^{(-)})&=-2\sqrt{2}<-2.\label{eq:a:sbell:psi-}
\end{align}
\end{subequations}

\subsection{Measurement of the S value}
In the experiment we measure the coincidence count rates for $\Psi^{(+)}$ and $\Psi^{(-)}$.
This is done by arranging the $\lambda/4$-plates in front of the polarizers in angles of $\alpha_{\lambda/4}=\beta_{\lambda/4}=\SI{0}{\degree}$ for $\Psi^{(+)}$ and $\alpha_{\lambda/4}=\SI{0}{\degree},\beta_{\lambda/4}=\SI{90}{\degree}$ for $\Psi^{(-)}$.
Now all necessary angle combinations according to \eqref{eq:a:angles} and equation \eqref{eq:a:EC} are tested.
The experimental results are listed in the tables \vref{tab:mesplus} and \vref{tab:mesminus}.

At first the mean value of the measured coincidence count rates $C_\text{mes}$ is calculated.
The standard deviation is used as the uncertainty:
\begin{align}
C
    &= \left\langle C_\text{mes}\right\rangle
    \label{eq:a:Cmean}\\
\Delta C
    &= \sqrt{\left\langle \left(C_\text{mes}-C\right)^2\right\rangle}
    \label{eq:a:dC}\\
C\left(\Psi^{(+)},\alpha=\SI{22.5}{\degree},\beta=\SI{0}{\degree}\right)
    &=\frac{170+172+166+165+161}{5}\;\si{\counts\per\second}
    \nonumber\\
    &=\SI{166.8}{\counts\per\second}
\nonumber\\
\Delta C\left(\Psi^{(+)},\alpha=\SI{22.5}{\degree},\beta=\SI{0}{\degree}\right)
    &=\sqrt{\frac{170^2+172^2+166^2+165^2+161^2}{5}-166.8^2}\;\si{\counts\per\second}
    \nonumber\\
    &=\SI{3.9}{\counts\per\second}
    \nonumber
\end{align}
With this values one can use equation \eqref{eq:a:EC} to calculate the expectation value
\begin{align*}
E\left(\Psi^{(+)},\alpha=\SI{22.5}{\degree},\beta=\SI{0}{\degree}\right)
    &=\frac{C(\alpha,\beta)-C(\alpha,\beta^\perp)-C(\alpha^\perp,\beta)+C(\alpha^\perp,\beta^\perp)}{C(\alpha,\beta)+C(\alpha,\beta^\perp)+C(\alpha^\perp,\beta)+C(\alpha^\perp,\beta^\perp)}
    \\
    &=\frac{166.8-619-638+176}{166.8+619+638+176}\;\frac{\si{\counts\per\second}}{\si{\counts\per\second}}
    \\
    &=-0.57
\end{align*}
with error propagation
\begin{subequations}
\label{eq:a:dE}
\begin{align}
\Delta E
    &= \sum_{\substack{\alpha=\lbrace\alpha,\alpha^\perp\rbrace\\\beta=\lbrace\beta,\beta^\perp\rbrace}}\abs{\frac{\dd E}{\dd C(\alpha,\beta)}}\Delta C(\alpha,\beta)
    \label{eq:a:dE:1}\\
    %&=\frac{\Delta C(\alpha,\beta)+\Delta C(\alpha,\beta^\perp)+\Delta C(\alpha^\perp,\beta)+\Delta C(\alpha^\perp,\beta^\perp)}{C(\alpha,\beta)+C(\alpha,\beta^\perp)+C(\alpha^\perp,\beta)+C(\alpha^\perp,\beta^\perp)}\left(1+\abs{E}\right)
    &=\left(1+\abs{E}\right)\frac{\sum_{\alpha,\beta}\Delta C(\alpha,\beta)}{\sum_{\alpha,\beta}C(\alpha,\beta)}
    \label{eq:a:dE:2}\\
\Delta E\left(\Psi^{(+)},\alpha=\SI{22.5}{\degree},\beta=\SI{0}{\degree}\right)
    &=(1+0.57)\frac{3.9+7.5+7.7+2.1}{166.8+619+638+176}\;\frac{\si{\counts\per\second}}{\si{\counts\per\second}}
    \nonumber\\
    &=0.02.
    \nonumber
\end{align}
\end{subequations}
Finally one can calculate the $S$ value using equation \eqref{eq:a:S} like
\begin{align*}
S\left(\Psi^{(+)}\right)
    &=E\left(\SI{22.5}{\degree},\SI{0}{\degree}\right)-E\left(\SI{22.5}{\degree},\SI{-45}{\degree}\right)+E\left(\SI{-22.5}{\degree},\SI{0}{\degree}\right)+E\left(\SI{-22.5}{\degree},\SI{-45}{\degree}\right)
    \\
    &=-0.57+0.63-0.64+0.52
    \\
    &=-0.06
\end{align*}
with error propagation
\begin{subequations}
\label{eq:a:dS}
\begin{align}
\Delta S
    &= \sum_{\substack{\alpha=\lbrace\alpha,\alpha'\rbrace\\\beta=\lbrace\beta,\beta'\rbrace}}\abs{\frac{\dd S}{\dd E(\alpha,\beta)}}\Delta E(\alpha,\beta)
    \label{eq:a:dS:1}\\
    &=\sum_{\alpha,\beta}\Delta E(\alpha,\beta)
    \label{eq:a:dS:2}\\
\Delta S\left(\Psi^{(+)}\right)
    &=0.02+0.04+0.02+0.02
    \nonumber\\
    &=0.10.
    \nonumber
\end{align}
\end{subequations}
The final results are:
\begin{align*}
S\left(\Psi^{(+)}\right)
    &=-0.06\pm 0.10
    \\
S\left(\Psi^{(-)}\right)
    &=-2.07\pm 0.09
\end{align*}
The $S$ value for $\Psi^{(+)}$ is 0 like expected but the $S$ value for $\Psi^{(-)}$ is much smaller than $2\sqrt{2}=2.83$.
In fact the value can be seen as possibly bigger than -2 within its uncertainty and is therefore no proof for the wrongness of the LHV theory.

\subsection{Tomography measurement of the density of states}
The density matrix of the emitted light is measured in this section.
In order to understand the procedure we look back at the formalism.
The density matrix is an operator and
a $4\cross 4$ shaped matrix in the chosen formalism.
This matrix is hermitian with a trace of 1 which implies that it has fifteen degrees of freedom.
One can now choose sixteen pure two photon states $\ket{k}=\ket{1}_k\otimes\ket{2}_k$, using combinations of $\state{H}$, $\state{V}$, $\state{D}$, $\state{R}$ and $\state{L}$, which differ in more than only a complex phase and measure their probability $P_k$.
The density matrix is then defined as
\begin{align}
\rho
    &=\sum_{k=1}^{16}P_k\ket{k}\bra{k}.
    \label{eq:a:rho}
\end{align}
The probability can be calculated from the coincidence count rates $C_k$ like
\begin{align}
P_k
    &=C_k \left(\sum_{l=1}^{16}C_l\right)^{-1},
    \label{eq:a:Pk}
\end{align}
with uncertainty
\begin{subequations}
\label{eq:a:dP}
\begin{align}
\Delta P_k
    &=\sum_{l=1}^{16}\abs{\frac{\dd P_k}{\dd C_l}}\Delta C_l
    \label{eq:a:dP:1}\\
    &=\frac{\Delta C_k+P_k\sum_{l=1}^{16}\Delta C_l}{\sum_{l=1}^{16}C_l}.
    \label{eq:a:dP:2}
\end{align}
\end{subequations}
The $\ket{k}\bra{k}$ term can be calculated with a matrix multiplication; for example $\state{DR}\brastate{DR}$:
\begin{subequations}
\label{eq:a:DR}
\begin{align}
\state{DR}\brastate{DR}
    &=\frac{1}{\sqrt{2}}\left(\state{H}+\state{V}\right) \otimes \frac{1}{\sqrt{2}}\left(\state{H}-i\state{V}\right) \brastate{DR}
    \label{eq:a:DR:1}\\
    &=\frac{1}{2}\left(\state{HH}-i\state{VV}-i\state{HV}+\state{VH}\right) \brastate{DR}
    \label{eq:a:DR:2}\\
    &=\frac{1}{4}
    \begin{pmatrix}1\\-i\\-i\\1\end{pmatrix}
    \begin{pmatrix}1&i&i&1\end{pmatrix}
    \label{eq:a:DR:3}\\
    &=\frac{1}{4}\begin{pmatrix}
    1   &i  &i  &1\\
    -i  &1  &1  &-i\\
    -i  &1  &1  &-i\\
    1   &i  &i  &1
    \end{pmatrix}
\end{align}
\end{subequations}
An uncertainty can be assumed using error propagation while splitting the real and the complex part
\begin{subequations}
\label{eq:a:drho}
\begin{align}
\Delta \rho
    &=\sum_{k=1}^{16}\abs{\frac{\dd\Re{\rho}}{\dd P_k}}\Delta P_k+i\sum_{k=1}^{16}\abs{\frac{\dd\Im{\rho}}{\dd P_k}}\Delta P_k
    \label{eq:a:drho:1}\\
    &=\sum_{k=1}^{16}\left(\abs{\Re{\ketbra{k}}}+i\abs{\Im{\ketbra{k}}}\right)\Delta P_k
    \label{eq:a:drho:2}
\end{align}
\end{subequations}
At first the states $\ket{k}$ must be determined.
A good choice could be
\begin{align*}
\state{HH}
    &=\begin{pmatrix} 1\\ 0\\ 0\\ 0\end{pmatrix}
    &\state{HV}
    &=\begin{pmatrix} 0\\ 0\\ 1\\ 0\end{pmatrix}
    &\state{VV}
    &=\begin{pmatrix} 0\\ 1\\ 0\\ 0\end{pmatrix}
    &\state{VH}
    &=\begin{pmatrix} 0\\ 0\\ 0\\ 1\end{pmatrix}
    \\
\state{RH}
    &=\frac{1}{\sqrt{2}}\begin{pmatrix} 1\\ 0\\ 0\\-i\end{pmatrix}
    &\state{RV}
    &=\frac{1}{\sqrt{2}}\begin{pmatrix} 0\\ 1\\-i\\ 0\end{pmatrix}
    &\state{DV}
    &=\frac{1}{\sqrt{2}}\begin{pmatrix} 0\\ 1\\ 1\\ 0\end{pmatrix}
    &\state{DH}
    &=\frac{1}{\sqrt{2}}\begin{pmatrix} 1\\ 0\\ 0\\ 1\end{pmatrix}
    \\
\state{DR}
    &=\frac{1}{2}\begin{pmatrix} 1\\-i\\-i\\ 1\end{pmatrix}
    &\state{DD}
    &=\frac{1}{2}\begin{pmatrix} 1\\ 1\\ 1\\ 1\end{pmatrix}
    &\state{RD}
    &=\frac{1}{2}\begin{pmatrix} 1\\-i\\ 1\\-i\end{pmatrix}
    &\state{HD}
    &=\frac{1}{\sqrt{2}}\begin{pmatrix} 1\\ 0\\ 1\\ 0\end{pmatrix}
    \\
\state{VD}
    &=\frac{1}{\sqrt{2}}\begin{pmatrix} 1\\ i\\ 1\\ i\end{pmatrix}
    &\state{VL}
    &=\frac{1}{\sqrt{2}}\begin{pmatrix} 0\\ i\\ 0\\ 1\end{pmatrix}
    &\state{HL}
    &=\frac{1}{\sqrt{2}}\begin{pmatrix} 1\\ 0\\ i\\ 0\end{pmatrix}
    &\state{RL}
    &=\frac{1}{2}\begin{pmatrix} 1\\ 1\\ i\\-i\end{pmatrix}
\end{align*}
which are listed with their representation in the actual formalism.
The corresponding polarizer and $\lambda/4$-plate angles so as the measured coincidence count rates are listed in table \vref{tab:mestom}.
The density matrix is calculated to
\begin{align*}
\rho
    &=\rhonormal\\
    &\pm \drhonormal .
\end{align*}
But what is it telling us?
We can represent the matrix in the basis of the bell states
\begin{align}
\ket{\Phi^{+}}
    &=\begin{pmatrix}1\\0\\0\\0\end{pmatrix}
    &\ket{\Phi^{-}}
    &=\begin{pmatrix}0\\1\\0\\0\end{pmatrix}
    &\ket{\Psi^{+}}
    &=\begin{pmatrix}0\\0\\1\\0\end{pmatrix}
    &\ket{\Psi^{-}}
    &=\begin{pmatrix}0\\0\\0\\1\end{pmatrix}
    \label{eq:a:bell_basis}
\end{align}
by multiplying the transformation matrix $B$
\begin{align}
\rho_\text{bell}
    &=B^\dagger \rho B
    \label{eq:a:rho_bell}\\
    &=\begin{pmatrix}
    1&1&0&0\\
    1&-1&0&0\\
    0&0&1&1\\
    0&0&1&-1\\
    \end{pmatrix}
    \rho
    \begin{pmatrix}
    1&1&0&0\\
    1&-1&0&0\\
    0&0&1&1\\
    0&0&1&-1\\
    \end{pmatrix}
    \nonumber\\
    &=\rhobell
    \nonumber\\
    &\pm \drhobell .
    \nonumber
\end{align}
One can read the probability to find the corresponding bell state from the diagonal of the density matrix.
At last the eigenvectors $v_i$ and eigenvalues $\lambda_i$ of the density matrix are calculated (in the previous formalism, not the bell basis).
\begin{align*}
\lambda_1
    &=\evala
    &v_1
    &=\eveca
    \\
\lambda_2
    &=\evalb
    &v_2
    &=\evecb
    \\
\lambda_3
    &=\evalc
    &v_3
    &=\evecc \approx \ket{\Phi^{(-)}}
    \\
\lambda_4
    &=\evald
    &v_4
    &=\evecd \approx \ket{\Psi^{(-)}}
\end{align*}
The eigenvectors $v_3$ and $v_4$ are close to the minus bell states.
