\section{Appendix}
You will find the tables with the original measurement data in the following.

\begin{table}[htbp]
\centering
\caption{
    Measurements for $\Psi^{+}$.
    The angles $\alpha$ and $\beta$ are the angles at the two polarizers.
    The coincidence count rates $C(\alpha,\beta)$ are taken five times as average over \SI{10}{\second} in order to reduce noise.
}
\label{tab:mesplus}
\bigtableplus
\end{table}

\begin{table}[htbp]
\centering
\caption{
    Measurements for $\Psi^{-}$.
    The angles $\alpha$ and $\beta$ are the angles at the two polarizers.
    The coincidence count rates $C(\alpha,\beta)$ are taken five times as average over \SI{10}{\second} in order to reduce noise.
}
\label{tab:mesminus}
\bigtableminus
\end{table}

\begin{table}[htbp]
\centering
\caption{
    Measurements for the tomography.
    The states S1 and S2 of photon 1 and 2 are choosen to provide the calculation of the density matrix of the light.
    The angles of the polarizers P1 and P2 and the $\lambda/4$ plates L1 and L2 for photon 1 and 2 must be choosen as mentioned.
    The coincidence count rates $C(\alpha,\beta)$ are taken five times as average over \SI{10}{\second} in order to reduce noise.
}
\label{tab:mestom}
\bigtabletom
\end{table}