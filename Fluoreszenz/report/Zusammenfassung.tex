\section{Summary}
To conclude, the CHSH inequality \eqref{eq:a:CHSH} yields for the LHV theory which states that the outcome of a quantum mechanic experiment is predetermined by a local hidden variable (LHV).
In this case the absolute of the variable $S$ \eqref{eq:a:S} must be smaller than 2.
In the report a theoretical description of the quantum mechanics without LHV are done first.
It can be shown that for the bell state $\Psi^{(-)}$ \eqref{eq:bell_states} the variable shall become $S=-2\sqrt{2}$.
The measurement of the $S$ value for two of the bell states showed $S(\Psi^{(+)})=-2.36\pm0.10$ and $S(\Psi^{(-)})=-2.07\pm0.09$.
The value for $\Psi^{(+)}$ violates the CHSH inequality and therefore proofs the wrongness of the LHV theory.
Nevertheless the expected value of $-2\sqrt{2}=-2.83$ is not achieved.
A possible reason may be background light or a not perfect polarization which causes the system to be not in the pure $\Psi^{(+)}$ state.
At last the density matrix $\rho$ of the generated $\Psi^{(+)}$ was determined by using a tomography measurement.
In the basis of the bell states
\begin{align*}
\ket{\Phi^{+}}
    &=\begin{pmatrix}1\\0\\0\\0\end{pmatrix}
    &\ket{\Phi^{-}}
    &=\begin{pmatrix}0\\1\\0\\0\end{pmatrix}
    &\ket{\Psi^{+}}
    &=\begin{pmatrix}0\\0\\1\\0\end{pmatrix}
    &\ket{\Psi^{-}}
    &=\begin{pmatrix}0\\0\\0\\1\end{pmatrix}
\end{align*}
it becomes
\begin{align*}
\rho
    &=\rhobell
    \\
    &\pm \drhobell .
\end{align*}
The diagonal contains the probabilities to find the system in the corresponding bell state.
The accordance of $\Psi^{(+)}$ is therefore only $0.73\pm0.04$.
A $S$ value can be calculated from the density matrix to $S=-1.87\pm 0.21$.