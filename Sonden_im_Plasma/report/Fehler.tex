\section{Unsicherheitsbetrachtung}
Der gr\"o\ss te Unsicherheitsfaktor in der Messung ist das veraltete Barometer.
Es zeigt Abweichungen von Teils sogar mehr als dem angegebenen Fehler $\Delta p=\sidp$.
Dies macht es quasi unm\"oglich einen pr\"azisen Druck einzustellen.
Die Messung der Str\"ome und Spannungen hingegen ist ziemlich genau.
Aufgrund der Unsicherheit der Messger\"ate sowie angenommener statistischer Fehler werden hier Fehler von $\Delta I=\sidI$ und $\Delta U=\sidU$ angenommen.
Dar\"uber hinaus k\"onnen die Fehler f\"ur Fitparameter direkt aus der Diagonalen ihrer Kovarianzmatrix berechnet werden.
F\"ur die Fits wird die Methode der kleinsten Quadrate unter Ber\"ucksichtigung des Fehlers von $I$ verwendet.

Der Fehler abgeleiteter Gr\"o\ss en wird mittels Fehlerfortpflanzung bestimmt.
Wenn $X$ von $U$ und $I$ abh\"angt dann ist der fehler von $X$
\begin{equation}
\Delta X = \abs{\frac{\partial X}{\partial U}}\Delta U +  \abs{\frac{\partial X}{\partial I}}\Delta I
\end{equation}
Weitere Fehlerquellen sind Temperaturschwankungen im Plasma und die M\"oglichkeit, dass der Elektronenstrom in die S\"attigung \"ubergeht und das Plasma um sich herum ver\"andert.
Letzteres ist an einem st an einem helleren Gl\"uhen um die Sonde herum zu erkennen und kann daher vermieden werden.

\section{Zusammenfassung}
Im Versuch werden die Kennlinien einer Langmuir- und einer Doppel-Sonde im Argon-Plasma untersucht.
Dabei wird die Langmuir-Sonde mit einer negativen Spannung betrieben.
Dies f\"uhrt dazu, dass die Sonde negativ geladen ist und Elektronen im Plasma abst\"o\ss t.
Die Kennlinie zeigt dabei zun\"achst eine S\"attigung bei stark negativen Spannungen.
Mit steigender Spannung ergibt sich zun\"achst ein exponentieller Anstieg des Stroms, der bald in eine lineare Steigung \"ubergeht.
F\"ur die Doppelsonde verl\"auft die Kurve im Elektronenanlaufbereich linear.
In den Elektronenstroms\"attigungsbereichen verl\"auft die Kurve ebenfalls linear, allerdings mit einer geringeren Steigung.
Der lineare Anteil in der S\"attigung kommt dabei von der zylindrischen Form des Sondenkopfes.

F\"ur beide Sonden werden die Elektronentemperatur $T_e$, die Elektronendichte $n_e$, der Ionisationsgrad $\alpha$, die Debye-L\"ange $\lambda_\text{D}$, sowie die Plasmafrequenzen f\"ur Elektronen $\omega_e$ und f\"ur Ionen $\omega_i$ untersucht.
Es zeigen sich dabei keine klaren Trends in Abh\"angigkeit des Gasdrucks $p$, was haupts\"achlich dem Ungenauen Barometer geschuldet ist.
Allerdings stellt sich heraus, dass $n_e$, $\omega_e$ und $\omega_i$ mit steigendem $p$ steigen, wh\"ahrend die anderen Werte fallen.
Elektronen Schwingen aufgrund ihrer geringeren Masse schneller.
Darum ist $\omega_e$ um etwa drei Gr\"o\ss enordnungen h\"oher als $\omega_i$.
%Der Ionisationsgrad sinkt leicht, was darauf schlie\ss en l\"asst, dass haupts\"achlich unselbstst\"andige Entladung vorliegt, die durch die Anwesenheit von mehr dichteren Atomen nicht beg\"unstigt wird.
Der abnehmende Ionisationsgrad kann darauf zur\"uckgef\'uhrt werden, dass die Durchschlagsspannung mit steigendem Druck steigt (Paschen-Beziehung).
Die Anwesenheit von mehr potentiell ionisierbaren Atomen verleitet dennoch zu einer h\"oheren Anzahl an freien Elektronen und Ionen.
Diese wiederum schirmen Elektrische Felder st\"arker ab, was die Debye-L\"ange verkleinert.
Die Plasmafrequenzen steigen ebenfalls aufgrund der h\"oheren Ladungstr\"agerdichte.
Die Elektronentemperatur hingegen sinkt.
H\"aufigere St\"o\ss e f\"uhren zu geringeren mittleren Wegstrecken, auf denen ein Elektron beschl\"aunigt werden kann, bevor es St\"o\ss t.

Au\ss erdem wird die Abh\"angigkeit der genannten Gr\"o\ss en von der radialen Position der Sonde im Plasma untersucht.
Dabei zeigt sich leider kein Trend, was allerdings wohl darauf zur\"uckzuf\"uhren ist, dass die mit der Messung abgedeckten Bereiche zu klein sind.

Zuletzt wird die Abh\"angigkeit vom Entladungsstrom $I_\text{E}$ untersucht.
Es zeigt sich, dass $n_e$, $\alpha$, $\omega_e$ und $\omega_i$ mit steigendem $I_\text{E}$ steigen; die anderen Werte fallen.
Der h\"ohere Strom (bei h\"oherer Spannung) f\"uhrt zu mehr Ionisation, was Ionisationsgrad und Elektronendichte erh\"oht.
Dies wiederum verst\"arkt die elektrische Abschirmung und senkt die Debye-L\"ange, w\"ahrend die Plasmafrequenzen steigen.
Wieder sinkt die Elektronentemperatur aufgrund h\"aufigerer St\"o\ss e.