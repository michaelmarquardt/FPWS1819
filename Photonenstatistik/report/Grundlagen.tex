\section{Grundlagen}
\subsection{Photonenquellen}
\begin{figure}[htbp]
    \begin{minipage}[t][][b]{0.6\textwidth}
        
        \caption{
            Ein 4-Niveau-System wird konstant extern von Zustand 1 in Zustand 2 angeregt.
            Dies geschieht zum Beispiel mit Licht der Energie $\hbar \nu_\text{x}$.
            Das System f\"allt durch Emission eines Phonons in Zustand 3 onhe Licht zu emittieren.
            Der \"Ubergang von 3 nach 4 setzt ein einzelnes Photon frei.
            Der \"Ubergang von 4 nach 1 l\"auft wieder durch Phononen.
        }
        \label{fig:4niv}
    \end{minipage}
    \hfill
    \begin{minipage}[t][][b]{0.38\textwidth}
        \flushright
        \begin{tikzpicture}[
        x=0.2\textwidth,
        y=0.15\textwidth,
        >=triangle 60,
        ]
        %\draw (0,0)rectangle(5,6);
        \draw[->] (0,0)--(0,5) node[right]{E};
        % states
        \draw
        (1,0.5)node[left]{1}--(2,0.5)
        (1,4.5)node[left]{2}--(2,4.5)
        (2,1)--(3,1)node[right]{4}
        (2,4)--(3,4)node[right]{3}
        ;
        % light
        \draw [
        ->,
        decorate,
        decoration={
            snake,
            amplitude=0.7mm,
            segment length=2.2mm,
            post length=2mm,
        },
        color=red,
        ]
        (0.1,1.5)--(1.45,2.5);
        \draw[color=red,->]
        (1.5,.5)--(1.5,4.5);
        \draw[color=red]
        (0.3,2.4)node[right]{$\hbar \nu_\text{x}$};
        \draw[->,color=grey]
        (1.8,4.5)--(2.2,4);
        \draw[->,color=blue]
        (2.5,4)--(2.5,1);
        \draw [
        ->,
        decorate,
        decoration={
            snake,
            amplitude=0.7mm,
            segment length=2.2mm,
            post length=2mm,
        },
        color=blue,
        ]
        (2.6,2.6)--(3.9,3.6);
        \draw[color=blue]
        (3,2.6)node[right]{$\hbar \nu_\text{e}$};
        \draw[->,color=grey]
        (2.2,1)--(1.8,0.5);
        \end{tikzpicture}
    \end{minipage}
\end{figure}
Eine Lichtquelle funktioniert prinzipiell dadurch, dass elektronische Zust\"ande in der Quelle angeregt werden.
Diese fallen dann in den Grundzustand zur\"uck, wobei sie ein Photon emittieren.
Ein Beispiel f\"ur einen einfachen \"Ubergang ist ein 4-Niveau-System wie in Abbildung \vref{fig:4niv}.
In makroskopischen Lichtquellen werden viele solche Systeme gleichzeitig angeregt.
F\"ur eine Einzel-Photon-Quelle hingegen darf nur ein Zustand angeregt werden.




\subsection{Moden}




\subsection{Bunching und Antibunching}





Test\cite{brouri}