\section{Grundlagen}
\subsection{Photonenquellen}\label{sec:Photonenquellen}
Eine Lichtquelle funktioniert prinzipiell dadurch, dass elektronische Zust\"ande in der Quelle angeregt werden.
Diese fallen dann in den Grundzustand zur\"uck, wobei sie ein Photon emittieren.
Ein Beispiel f\"ur einen einfachen \"Ubergang ist ein 4-Niveau-System, wie in Abbildung \vref{fig:4niv} dargestellt.
In makroskopischen Lichtquellen werden viele solche Systeme gleichzeitig angeregt.
F\"ur eine Einzelphotonenquelle hingegen darf nur ein Zustand angeregt werden. 
Das Zurückfallen in den Grundzustand passiert nicht sofort nach der Anregung, sondern nach einer mittleren Lebensdauer. Bei einem NV-Zentrum beträgt diese $\tau = 11.6\ \mathrm{ns}$ \cite{brouri}. Dies bedeutet nicht, dass immer das Photon genau nach dieser Zeitspanne emittiert wird, sondern es handelt sich um einen zufälligen Prozess. Lediglich im Mittel werden die Photonen nach dieser Zeitspanne emmitiert. 


\begin{figure}[htbp]
    \begin{minipage}[t][][b]{0.6\textwidth}
        
        \caption{
            Ein 4-Niveau-System wird konstant extern von Zustand 1 in Zustand 2 angeregt.
            Dies geschieht zum Beispiel mit Licht der Energie $\hbar \nu_\text{x}$.
            Das System f\"allt durch Emission eines Phonons in Zustand 3 onhe Licht zu emittieren.
            Der \"Ubergang von 3 nach 4 setzt ein einzelnes Photon frei.
            Der \"Ubergang von 4 nach 1 l\"auft wieder durch Phononen.
        }
        \label{fig:4niv}
    \end{minipage}
    \hfill
    \begin{minipage}[t][][b]{0.38\textwidth}
        \flushright
        \begin{tikzpicture}[
        x=0.2\textwidth,
        y=0.15\textwidth,
        >=triangle 60,
        ]
        %\draw (0,0)rectangle(5,6);
        \draw[->] (0,0)--(0,5) node[right]{E};
        % states
        \draw
        (1,0.5)node[left]{1}--(2,0.5)
        (1,4.5)node[left]{2}--(2,4.5)
        (2,1)--(3,1)node[right]{4}
        (2,4)--(3,4)node[right]{3}
        ;
        % light
        \draw [
        ->,
        decorate,
        decoration={
            snake,
            amplitude=0.7mm,
            segment length=2.2mm,
            post length=2mm,
        },
        color=red,
        ]
        (0.1,1.5)--(1.45,2.5);
        \draw[color=red,->]
        (1.5,.5)--(1.5,4.5);
        \draw[color=red]
        (0.3,2.4)node[right]{$\hbar \nu_\text{x}$};
        \draw[->,color=grey]
        (1.8,4.5)--(2.2,4);
        \draw[->,color=blue]
        (2.5,4)--(2.5,1);
        \draw [
        ->,
        decorate,
        decoration={
            snake,
            amplitude=0.7mm,
            segment length=2.2mm,
            post length=2mm,
        },
        color=blue,
        ]
        (2.6,2.6)--(3.9,3.6);
        \draw[color=blue]
        (3,2.6)node[right]{$\hbar \nu_\text{e}$};
        \draw[->,color=grey]
        (2.2,1)--(1.8,0.5);
        \end{tikzpicture}
    \end{minipage}
\end{figure}





\subsection{Bunching und Antibunching}\label{sec:Bunching}
Unter Antibunching versteht man das nicht auftreten von zeitlicher Korrelation einzelner Photonen aus derselben Quelle. Dabei handelt es sich um einen quantenmechanischen Effekt, der bei klassischen Lichtquellen nicht auftritt.  Deshalb eignet sich die Untersuchung auf zeitliche Korrelation als Nachweis für nicht klassische Lichtquellen. 
Eine Möglichkeit diese Korrelation experimentell zu untersuchen, ist die Verwendung des Hanbury Brown-Twiss Effekts \cite{brouri}. Der schematische Aufbau eines Experiments zur Messung des Hanbury Brown-Twiss Effekts ist in der Abbildung \ref{fig:Versuchsaufbau} dargestellt. Das Grundprinzip ist, dass die emmitierten Photonen einer Lichtquelle mittels eines Strahlteilers auf zwei Detektoren verteilt werden. Bei einer klassischen Lichtquelle, zum Beispiel einer thermischen Lichtquellen, können die beiden Detektoren gleichzeitig Photonen aus derselben Quelle messen. Die Ursache dafür ist, dass wie in Kapitel \ref{sec:Photonenquellen} beschrieben, viele Anregungen gleichzeitig stattfinden und somit auch mehrere Photonen gleichzeitig emmitiert werden können. Dabei tritt Korrelation auf, was als Bunching bezeichnet wird. Bei einer nicht klassischen Lichtquelle, ist dies nicht möglich. Als Ergebnis der Messung erhält man ein Histogramm mit den Zeitdifferenzen zwischen dem Eintreffen eines Photons an einem Detektor und eines anderen Photons an dem zweiten  Detektor. 

\subsection{NV-Zentren}
Bei diesem Versuch soll als nicht klassische Lichtquelle ein NV-Zentrum untersucht werden. Ein NV-Zentrum ist eine Störstelle in einem Diamanten, bei der zwei benachbarte Kohlenstoffatome durch eine Fehlstelle und ein Stickstoffatom ersetzt worden sind.  
Hergestellt werden können NV-Zentren durch Bestrahlung eines möglichst reinen Diamanten mit Stickstoffatomen. Diese Methode ist gut steuerbar und deshalb ist es möglich einzelne Defekte zu erzeugen. Durch den Beschuss mit den Stickstoffatomen werden außerdem Bindungen im Diamanten gebrochen und es entstehen unbesetzte Gitterplätze. Durch Erhitzen des Diamanten wandern die Fehlstellen und die Fremdatome, hier die Stickstoffatome, zu einander.  
In einem NV-Zentrum kann es zu Anregungen kommen, wie sie im Kapitel \ref{sec:Photonenquellen} beschrieben sind. Da es pro NV-Zentrum nur einen anregbaren Zustand gibt, handelt es sich bei einem NV-Zentrum um eine Einzelphotonenquelle.   Bei einer Einzelphotonenquelle wird immer nur ein Photon emmitiert. Danach muss erst eine erneute Anregung durchgeführt werden, bevor ein zweites Photon emittiert werden kann. Deshalb können an den Detektoren niemals zwei Photonen gleichzeitig gemessen werden. NV-Zentren sind somit ein Beispiel für ein System, bei dem Antibunching, siehe Kapitel \ref{sec:Bunching}  auftritt.  
Die emittierten Photonen haben alle unterschiedliche Moden. Ein Mode hat  einen speziellen Wellenvektor $\vec{k}$, eine bestimmte Polarisierung und eine bestimmte Frequenz. Für die Messsung der Korrelation ist es notwendig, dass nur Photonen einer bestimmten Mode  auf die beiden Detektoren treffen.




