\section{Zusammenfassung}
Im Versuch wurde die Photonenstatistik von NV-Zentren und klassischen Lichtquellen untersucht.
Bei NV-Zentren handelt es sich um St"orstellen in einem Diamanten.
Die Bandstruktur eines NV-Zentrums erlaubt es einzelne Photonen zu erzeugen.
An zwei Detektoren im selben Abstand zur Lichtquelle (Interferometer) werden die Klicks durch ankommende Photonen gemessen und eine Korrelationsfunktion $g^{(2)}(\tau)$ berechnet, wobei $\tau$ die Zeit zwischen einem Klick an Detektor 1 und dem n"achsten Klick an Detektor 2 ist.
Bei einem einzelnen NV-Zentrum sollten keine zwei Photonen zur selben Zeit erzeugt werden, was einem Wert $g^{(2)}(0)=0$ entspr"ache.
Die Messung allerdings zeigt Werte um die $g^{(2)}(0)\approx 0.77$.
Die Korrelationsfunktion steigt dann exponentiell auf 1 mit einer Korrelationszeit von \SI{25.1(18)}{\nano\second} in der ersten und \SI{5.5(6)}{\nano\second} in der zweiten Messung.

Anschlie\ss end wird noch der Hellste Punkt der Probe analysiert.
Hier sind die Photonen nahezu unkorreliert mit konstantem $g^{(2)}(\tau)=1$.
Die H"aufung von Photonen, also die Bevorzugte gleichzeitige Erzeugung von Photonen in klassischen Lichtquellen, wird jedoch nicht beobachtet.