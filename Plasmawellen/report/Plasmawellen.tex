% Alex header 
\documentclass[aps,amsmath,amssymb,amsfont]{revtex4-1}
\usepackage[utf8]{inputenc}
\usepackage[T1]{fontenc}
\usepackage[german]{babel}
\usepackage{float}
\usepackage{graphicx}
%\usepackage{siunitx}
\usepackage[colorlinks=true,linkcolor=black]{hyperref}
\usepackage{lmodern}
\usepackage{physics}
\usepackage{mathpazo}
\usepackage{here}
\usepackage{amsmath}
\usepackage{mathtools}
\usepackage[utf8]{inputenc}
\usepackage[T1]{fontenc}
\usepackage{lmodern}
\usepackage{subcaption}
\usepackage{placeins}
\usepackage{braket}
\numberwithin{equation}{section}
\usepackage{tikz}
\usetikzlibrary{arrows, calc}
\usepackage[europeanresistors,americaninductors]{circuitikz}
\makeatletter
\AtBeginDocument{
\def\tocname{Inhalt}
	\def\andname{and}
	\def\Dated@name{Dated: }
	\def\figurename{Abb.}
	\def\tablename{Tab.}
}
\makeatother


\newcommand{\subfig}[1]{\begin{subfigure}[t]{0.49\textwidth}
        \centering
        \includegraphics[width=1.1\textwidth]{#1}
        \caption{}	
    \end{subfigure}}
\newcommand{\pictures}[1]{\begin{subfigure}[t]{0.49\textwidth}
        \centering
        \includegraphics[width=1.1\textwidth]{#1}
        \caption{}
    \end{subfigure}}


%Michael header
%\documentclass[
%    captions=nooneline
%    ,headinclude
%    ,headsepline
%    ,bibliography=totocnumbered
%]{scrartcl}
%\usepackage[twoside,paper=a4paper,left=35mm,right=20mm,top=30mm,bottom=35mm]{geometry} 

%\usepackage[T1]{fontenc}
%\usepackage[utf8]{inputenc}
\usepackage[german]{varioref}

%%% GRAPHIKEN %%%
%\usepackage{graphicx}
\usepackage[labelfont=bf,justification=justified,format=plain]{caption}
\usepackage{subcaption}
\usepackage{placeins}
\usepackage{wrapfig}
%\usepackage{floatrow}
%% DREHUNG VON GRAFIKEN
%\usepackage{rotating}
%\usepackage{pdflscape}

%%% Tabellen %%%
%\usepackage{threeparttable}
%\usepackage{booktabs}
%\usepackage{multirow}

%%% MATHE %%%
%\usepackage{mathtools}
%\usepackage{physics}
%\usepackage{upgreek}
%\usepackage{amssymb}
\usepackage[separate-uncertainty=true,
            list-units = single,
            list-separator = {;\,},
            exponent-product=\cdot,%
            ]{siunitx}
%\DeclareSIUnit\molar{\mole\per\l}

%%% ZEICHENUMGEBUNG %%%
\usepackage{pgf}


%%% BIBLIOGRAPHIE %%%
%\usepackage[style=numeric-comp,backend=biber,sortlocale=de_DE,natbib=true]{biblatex}
%\addbibresource{Quellen.bib}
%\usepackage[babel,german]{csquotes}

%%% EIGENE DEFINITIONEN %%%
%\newcommand{\iu}{{i\mkern1mu}}
%\usepackage{txfonts} % |R, |C

%\usepackage{hyperref}

%%% INCLUDE VALUES %%%
%\newcommand{\bigtableplus}{
\begin{tabular}{S[table-format=+3.1]S[table-format=+2]|S[table-format=3]S[table-format=3]S[table-format=3]S[table-format=3]S[table-format=3]}
{$\alpha$ [\si{\degree}]}&{$\beta$ [\si{\degree}]}&\multicolumn{5}{c}{$C(\alpha,\beta)$ [counts/s]}\\\hline
22.5&0&170&172&166&165&161\\
22.5&90&624&627&616&622&606\\
112.5&0&644&648&632&639&627\\
112.5&90&176&177&179&178&173\\\hline
22.5&-45&158&157&145&151&154\\
22.5&45&628&632&656&641&631\\
112.5&-45&646&665&661&652&680\\
112.5&45&131&149&145&150&151\\\hline
-22.5&0&145&141&141&153&141\\
-22.5&90&599&593&592&577&593\\
67.5&0&640&623&637&623&634\\
67.5&90&123&121&120&127&121\\\hline
-22.5&-45&636&642&642&642&649\\
-22.5&45&166&171&168&174&183\\
67.5&-45&196&210&201&211&203\\
67.5&45&558&563&562&562&549\\\hline
\end{tabular}}
\newcommand{\bigtableminus}{
\begin{tabular}{S[table-format=+3.1]S[table-format=+2]|S[table-format=3]S[table-format=3]S[table-format=3]S[table-format=3]S[table-format=3]}
{$\alpha$ [\si{\degree}]}&{$\beta$ [\si{\degree}]}&\multicolumn{5}{c}{$C(\alpha,\beta)$ [counts/s]}\\\hline
22.5&0&84&91&89&84&88\\
22.5&90&634&629&630&624&615\\
112.5&0&625&618&622&624&625\\
112.5&90&79&73&76&80&75\\\hline
22.5&-45&441&448&453&450&448\\
22.5&45&267&265&260&264&272\\
112.5&-45&274&269&280&264&268\\
112.5&45&407&423&415&421&430\\\hline
-22.5&0&248&258&252&254&247\\
-22.5&90&448&459&441&450&444\\
67.5&0&535&514&502&503&514\\
67.5&90&218&211&213&215&219\\\hline
-22.5&-45&94&97&96&102&96\\
-22.5&45&616&627&607&630&611\\
67.5&-45&647&637&632&621&631\\
67.5&45&108&103&107&109&111\\\hline
\end{tabular}}


%\includeonly{}

\tikzset{
    declare function={% in case of CVS which switches the arguments of atan2
        atan3(\a,\b)=ifthenelse(atan2(0,1)==90, atan2(\a,\b), atan2(\b,\a));},
    kinky cross radius/.initial=+.125cm,
    @kinky cross/.initial=+, kinky crosses/.is choice,
    kinky crosses/left/.style={@kinky cross=-},kinky crosses/right/.style={@kinky cross=+},
    kinky cross/.style args={(#1)--(#2)}{
        to path={
            let \p{@kc@}=($(\tikztotarget)-(\tikztostart)$),
            \n{@kc@}={atan3(\p{@kc@})+180} in
            -- ($(intersection of \tikztostart--{\tikztotarget} and #1--#2)!%
            \pgfkeysvalueof{/tikz/kinky cross radius}!(\tikztostart)$)
            arc [ radius     =\pgfkeysvalueof{/tikz/kinky cross radius},
            start angle=\n{@kc@},
            delta angle=\pgfkeysvalueof{/tikz/@kinky cross}180 ]
            -- (\tikztotarget)}}}

\begin{document}
	\begin{titlepage}
	\centering
	\par\vspace{1cm}
	{\scshape\LARGE Universität Stuttgart \par}
	\vspace{1cm}
	{\scshape\Large  Fortgeschrittenen Praktikum \\ Wintersemester 2018/2019\par}
	\vspace{1.5cm}
	{\huge\bfseries Ionenschallwellen im Plasma\par}
	\vspace{2cm}
	{\Large\itshape Gruppe: M09\\ Alexander Sattler  \\  Michael Marquardt    \\ Versuchsdatum: Montag 17. Dezember 2018\\ Betreuer: Alf Köhn\\ \par}
	
	\end{titlepage}


\tableofcontents

%\siUerr
%\siIerr
\section{Grundlagen}
\subsection{Plasma}\label{sec:plasma}
Ein Gas, das teilweise oder vollständig ionisiert ist, wird als Plasma \cite{Plasmaphysik} bezeichnet. Bei der Ionisation dissoziieren die Gasteilchen in Elektronen und Ionen. Die Konzentration der beiden Ladungsarten ist ungefähr gleich groß. Deshalb ist ein Plasma nach außen hin neutral, obwohl es aus geladenen Teilchen besteht. Diese Eigenschaft wird als Quasineutralität bezeichnet. Damit verbunden ist die Eigenschaft, dass ein Plasma elektromagnetische Felder abschirmen kann.
Die Debye-Länge $\lambda_{\mathrm{D}}$ ist die Länge, nach der das Feld auf das $1/e$ fache seines ursprünglichen Wertes abgefallen ist. Sie ist definiert als
\begin{align}
\lambda_{\mathrm{D}}=\sqrt{\frac{\epsilon_0 T_{\mathrm{e}}}{e^2 n_\mathrm{e}}},
\end{align}
wobei $T_{\mathrm{e}}$ die Temperatur der Elektronen, $n_\mathrm{e}$ die Teilchenzahldichte der freien Elektronen, $e$ die Elementarladung und $\epsilon_0$ die Dielektrizitätskonstante ist.
Die Kugel mit der Debye-Länge als Radius wird als Debye-Kugel bezeichnet.
Bevor ein ionisiertes Gas als Plasma bezeichnet werden kann, gibt es Bedingungen die erfüllt sein müssen.  Eine Bedingung ist, dass die Ausdehnung des ionisierten Gases deutlich größer ist als die Debye-Länge. Zusätzlich muss es eine hinreichend große Anzahl Teilchen geben, die sich innerhalb der Debye-Kugel befinden. Des weiteren muss die Zeit zwischen den Stößen zweier Gasteilchen viel größer als die Periodendauer der Plasmaoszillation sein. Bei der Plasmaoszillation handelt es sich um eine periodische Oszillation der Ladungsdichte. Die Ursache dafür ist, dass die Elektronen in den Potentialen der positiven Ionen schwingen. Die Frequenz der Oszillation ist die Plasmafrequenz
\begin{align}
\omega_{\mathrm{p}}=\sqrt{\frac{n_{\mathrm{e}}e^2}{\epsilon_0 m_{\mathrm{e}}}},
\end{align}
wobei $m_{\mathrm{e}}$ die Elektronenmasse ist. 
In der Mitte der Kugel befindet sich ein geladenes Teilchen. Um dieses Teilchen befinden sich aufgrund der Coulomb-Wechselwirkung mehr entgegengesetzt geladenen Teilchen, als gleich geladene Teilchen. Folglich entsteht um das geladene Teilchen ein entgegengesetzt geladener Bereich. Dieser Bereich schirmt die Ladung nach außen hin ab.
Die Erzeugung eines Plasmas kann auf unterschiedliche Arten erfolgen. Bei einer selbstständige Gasentladung kommt es durch Anlegen einer genügend hohen Gleichspannung zu einem Ladungstransport, wie zum Beispiel bei einer Glimmentladung.  Der schematische Aufbau einer Glimmentladung und der Spannungsverlauf ist in der Abbildung \ref{fig:Glimmentladung} dargestellt.  Dabei wird an zwei Elektroden ein elektrisches Feld angelegt. Der Spannungsabfall zwischen der Kathode und dem negativem Glimmlicht wird als Kathodenfall bezeichnet. Durch das elektrische Feld wird den Gasteilchen Energie zugeführt und diese dadurch ionisiert.  Die ionisierten Gasteilchen haben dadurch ausreichend Energie,  um  durch Stöße weitere Gasteilchen zu  Ionisieren. Dieser Vorgang wird als Stoßionisation bezeichnet. Eine Stoßionisation ist eine selbständige Gasentladung, da dieser Vorgang unabhängig von äußeren Einflüssen stattfinden kann. Die durchschnittliche Strecke die ein Teilchen zwischen zwei Stößen zurücklegen kann, wird als mittlere freie Weglänge bezeichnet. 
\begin{figure}[H]
\centering
\includegraphics[scale=0.6]{Glimmentladung}
\caption{Schematischer Aufbau und einie Parameterverläufe einer Glimmentladung. Der Aufbau kann in verschiede Bereich unterteilt werden: Kathodenschicht (H1), negative Glimmzone (H2), positive Säule (H3) und Anodische Glühzone (H4), sowie Aston- (D1), Hittorf- (D2), Farraday- (D3) und Anodischer Dunkelraum (D4). Unterhalb des schematischen Aufbaus sind einige Parameterverläufe dargestellt: Leuchtstärke, Potential, elektrische Feldstärke, Strom durch Elektronen und Gesamtstrom (gestrichelt), Elektronen- (gestrichelt) und Ionendichte.   \cite{Plasmaphysik}}
\label{fig:Glimmentladung}
\end{figure}
Bei den Ladungsträgern in einem Plasma handelt es sich um Elektronen und positiv geladenen Ionen. Das leichteste Ion wäre ein einzelnes Proton. Das Massenverhältnis zwischen den unterschiedlichen Ladungsträgern unterscheidet sich somit deutlich, da ein Proton ungefähr 1800 mal schwerer ist als ein Elektron. Dieser Unterschied ist für die Eigenschaften eines Plasmas von großer Bedeutung und kann auch für die Untersuchung eines Plasmas benutzt werden. Darauf wird im Kapitel \ref{sec:Langmuir_Sonde} detaillierter eingegangen. Bei einem Plasma kann häufig eine Leuchterscheinung beobachtet werden. Diese wird verursacht durch Strahlungsemission  der angeregten Atome. Bei sehr hohen und bei sehr tiefen Temperaturen gibt es dies nicht. Bei sehr tiefen Temperaturen sind keine Atome in einem angeregtem Zustand und somit kann auch kein Elektron auf einen Zustand niedriger Energie gelangen. Bei sehr hohen Temperaturen sind die Atome vollständig ionisiert und es gibt keine Elektronen mehr, die einen Übergang zwischen unterschiedlichen Energieniveaus machen könnten. Wie stark ein Plasma ionisiert ist, kann durch den  Ionisationsgrad angegeben werden. Der Ionisationsgrad 
\begin{align}
  \alpha=\frac{n_{\mathrm{i}}}{n + n_{\mathrm{i}}}
\end{align}
ist null, im idealem neutralem Gas und eins im Fall einer vollständigen Ionisierung, wobei $n_{\mathrm{i}}$ die Dichte der freien Ladungsträger und $n$ die gesamte Teilchendichte ist. Der Abhängigkeit der Durchschlagspannung vom Gasdruck und der Schlagweite wird durch das Paschengesetz beschrieben. Unter der Durchschlagspannung versteht man die Spannung, ab der ein Isolator zu einem Plasma wird. 
\subsection{Langmuir-Sonde}\label{sec:Langmuir_Sonde}  
Eine Möglichkeit ein Plasma zu untersuchen ist die Verwendung von einer Langmuir-Sonde \cite{anleitung}. Damit können Eigenschaften des Plasmas wie Elektronendichte, Elektronentemperatur und Plasmapotential bestimmt werden. Die Schaltung zur Erzeugung eines Plasmas mit einer Glimmentladung und der Messung mit einer Langmuir-Sonde ist in der Abbildung \ref{fig:Schaltung_Sonde} dargestellt. 
\begin{figure}[H]
\centering
\includegraphics[scale=0.4]{Schaltung_Sonde.png}
\caption{Schaltung zur Erzeugung eines Plasmas mit einer Glimmentladung und der Messung mit einer Langmuir-Sonde. \cite{anleitung}}
\label{fig:Schaltung_Sonde}
\end{figure}
Die Spitze einer Langmuir-Sonde besteht aus einer Elektrode, die wiederum meistens aus einem dünnem Wolfram-Draht besteht. Die Sonde ist außer an der Spitze isoliert. Das Potential der Sonde unterscheidet sich von dem Potential des Plasmas. Dies kann durch das im Kapitel \ref{sec:plasma} erläuterte Massenverhältnis zwischen positiven und negativen Ladungsträger erklärt werden. Aufgrund ihrer geringeren Masse haben Elektronen eine höhere Mobilität. Bei gleicher Elektronen- und Ionentemperatur sind die Elektronen somit schneller als die Ionen. Dadurch treffen Elektronen häufiger auf die Sonde als die Ionen. Die Sonde lädt sich somit negativ auf. Durch das negative Aufladen der Sonde entsteht eine elektrostatische Kraft, die Elektronen abstößt und  Ionen anzieht. Im Gleichgewichtszustand wird die höhere Mobilität durch die elektrostatische Abstoßung kompensiert. Das heißt, die Elektronen- und Ionenströme sind gleich groß. Das Potential der Sonde das sich im Gleichgewichtsfall einstellt, wird als Floatingpotential $\Phi_{\mathrm{fl}}$ bezeichnet. In der Kennlinie \ref{fig:Kennlinie_Sonde} entspricht dies dem  Nulldurchgang.  Um den Rest der Kennlinie zu messen, muss eine Spannung an die Sonde angelegt werden. Wird die Sonde negativ aufgeladen, werden Elektronen abgestoßen und Ionen angezogen. Bei einer ausreichend starken negativen Aufladung, sind alle Ionen aus der Umgebung der Sonde abgezogen worden. Die Stromstärke steigt nun bei, zu negativen Werten hin abfallenden, Spannung nicht mehr an und wird als Ionensättigungstrom bezeichnet. Dies ist der Bereich im Diagramm \ref{fig:Kennlinie_Sonde}, bei dem ein konstanter positiver Strom gemessen werden kann. Der Ionensättigungsstrom kann mit
\begin{align}
 I_{\mathrm{i},\mathrm{sat}}=0.61enS \sqrt{\frac{T_\mathrm{e}}{m_{\mathrm{i}}}},
\end{align} 
berechnet werden, wobei $T_{\mathrm{e}}$ die Elektronentemperatur, $S$ die effektive Sondenoberfläche und $m_{\mathrm{i}}$ die Ionenmasse ist. Analog zum Ionensättigungsstrom kann auch ein Elektronensättigungsstrom gemessen werden. Dazu muss die Sonde ausreichen positiv geladen sein, damit alle Elektronen aus der Umgebung der Sonde von dieser angezogen worden sind. Im Idealfall einer unendlich ausgedehnten, ebenen Sonde wird der Elektronensättigungsstrom konstant. In dem Diagramm der Kennlinie ist dies der Bereich, mit dem konstantem negativem Strom. Berechnet werden kann der Elektronensättigungsstrom mit
\begin{align}
I_{\mathrm{e},\mathrm{sat}}=-enS \sqrt{\frac{T_{\mathrm{e}}}{2\pi m_{\mathrm{e}}}}.
\label{eq:Ionensättigungsstrom}
\end{align}
Der Idealfall lässt sich experimentell nicht realisieren, das heißt das die Stromstärke nicht konstant bleibt, wenn die Spannung steigt. An der Stelle, an der die reale Kennlinie größer wird als der Elektronensättigungsstrom, gibt es einen Knick in der Kennlinie. Die Stelle, an der der Knick ist, wird als Plasmapotential bezeichnet. Das heißt, der Elektronenlaufbereich endet am Plasmapotential. Der Verlauf der Kennlinie für größere Stromstärken als der Elektronensättigungsstrom, ist von der Geometrie der Sonde anhängig. Der Bereich zwischen den Sättigungsströmen wird Elektronenlaufbereich genannt. In diesem Bereich können nur die Elektronen die Sonde erreichen, die  ausreichend thermische Energie haben. Der Elektronenstrom im Elektronenlaufbereich ist gegeben durch
\begin{align}
I_{\mathrm{e}}=I_{\mathrm{e},\mathrm{sat}}\ \mathrm{exp} \left( -\frac{e(\Phi_{\mathrm{p}}-U)}{T_{\mathrm{e}}} \right),
\label{eq:Elektronenlaufbereich}
\end{align}
wobei $\Phi_{\mathrm{p}}$ das Plasmapotential und $U$ die angelegte Spannung an der Sonde ist. Der gesamte Strom im Elektronenlaufbereich ist die Summe aus Elektronenstrom und Ionenstrom
\begin{align}
I=I_{\mathrm{e}} +  I_{\mathrm{i},\mathrm{sat}}=  I_{\mathrm{e},\mathrm{sat}}\ \mathrm{exp} \left( -\frac{e(\Phi_{\mathrm{p}}-U)}{T_{\mathrm{e}}} \right) + I_{\mathrm{i},\mathrm{sat}}.
\end{align}
Dies lässt sich umformen in
\begin{align}
\mathrm{ln}(I_{\mathrm{i},\mathrm{sat}} -I) = \mathrm{ln}(I_{\mathrm{e},\mathrm{sat}}) - \frac{e(\Phi_{\mathrm{p}}-U)}{T_{\mathrm{e}}}.
\end{align}
Aus der Kennlinie kann der Ionensättigungsstrom abgelesen werden. Durch die Korrektur des Gesamtstromes durch den Ionensättigungsstrom, kann man im einem halb-logarithmischem Diagramm aus dem Offset die  Dichte und aus der Steigung die Elektronentemperatur bestimmen. 
\begin{figure}[H]
\centering
\includegraphics[scale=0.4]{Kennlinie_Sonde.png}
\caption{Kennlinie einer Langmuir-Sonde. \cite{anleitung}}
\label{fig:Kennlinie_Sonde}
\end{figure}
\subsection{Doppel-Sonde}
Für die Messung mit einer einzelnen Langmuir-Sonde wird ein Referenzpotential benötigt, welches jedoch nicht immer möglich ist (zum Beispiel auf einem Satelliten). In so einem Fall, kann eine Doppel-Sonde \cite{anleitung} verwendet werden, da dafür kein Referenzpotential notwendig ist. Die Schaltung für eine Doppelsonde ist in der Abbildung \ref{fig:Schaltung_Doppel_Sonde} dargestellt. In der Abbildung sieht man, dass der Strom dabei von der einen Sonde durch das Plasma zur anderen Sonde fließt. 

\begin{figure}[H]
\centering
\includegraphics[scale=0.4]{Schaltung_Doppel_Sonde.png}
\caption{Schaltung zur Erzeugung eines Plasmas mit einer Glimmentladung und der Messung mit einer Doppel-Sonde. \cite{anleitung}}
\label{fig:Schaltung_Doppel_Sonde}
\end{figure}
Für die Messung wird wie bei der Langmuir-Sonde die Spannung variiert und der Strom gemessen.  Die Kennlinie einer Doppelsonde ist in der Abbildung \ref{fig:Kennlinie_Doppel_Sonde} dargestellt. Wird an die Doppelsonde keine Spannung angelegt, fließt auch kein Strom. Das bedeutet der Nulldurchgang der Kennlinie ist im Ursprung.  In diesem Fall ist das System im Gleichgewicht und analog zur Langmuir-Sonde wird das Potential der beiden Sonden als Floatingpotential bezeichnet. Bei sehr großen Spannungen, sowohl negativ als auch positiv, geht der Strom in den Ionensättigungsbereich über. Der Verlauf der Kennlinie im Sättigungsbereich hängt von der Geometrie der beiden Sonden ab. Bei identischen Sonden, ist die Kennlinie punktsymmetrisch zum Ursprung. Zwischen den Bereichen mit der Elektronensättigung ist der Elektronenlaufbereich. 
\begin{figure}[H]
\centering
\includegraphics[scale=0.4]{Kennlinie_Doppel_Sonde.png}
\caption{Kennlinie einer Doppel-Sonde. \cite{anleitung}}
\label{fig:Kennlinie_Doppel_Sonde}
\end{figure}
Der gesamte Sondenstrom ist die Summe der Beiträge von den Elektronen und den Ionen gemäß
\begin{align}
I_{\mathrm{S}}=I_{\mathrm{e}1} -I_{\mathrm{i}1} = I_{\mathrm{i}2} -I_{\mathrm{e}2}.
\label{eq:Gesamter_strom}
\end{align} 
Für den Elektronenlaufbereich gilt mit \eqref{eq:Elektronenlaufbereich}
\begin{align}
I_{\mathrm{e}1} &=I_{\mathrm{e}1,\mathrm{sat}}\ \mathrm{exp} \left( -\frac{e(\Phi_{\mathrm{p}}-U_1)}{T_{\mathrm{e}}} \right) \label{eq:ELektronenlaufbereich_Doppel_sonde1} \\
I_{\mathrm{e}2} &=I_{\mathrm{e}2,\mathrm{sat}}\ \mathrm{exp} \left( -\frac{e(\Phi_{\mathrm{p}}-U_2)}{T_{\mathrm{e}}} \right). 
\label{eq:ELektronenlaufbereich_Doppel_sonde2}
\end{align}
Mit den Gleichung \eqref{eq:Gesamter_strom} und  \eqref{eq:Ionensättigungsstrom} und der Spannung $U=U_1-U_2$ lässt sich der Ausdruck
\begin{align}
\frac{I_{\mathrm{S}} + I_{\mathrm{i}1}}{ I_{\mathrm{i}2} - I_{\mathrm{S}}}= \frac{I_{\mathrm{e}1}}{I_{\mathrm{e}2}}= \frac{S_1}{S_2} \mathrm{exp} \left( \frac{e U_{\mathrm{S}}}{T_{\mathrm{e}}} \right)
\end{align}
herleiten. Aus der Annahme das $I_{\mathrm{i}1}$ und $I_{\mathrm{e}2}$ unabhängig von $U_{\mathrm{S}}$ sind, folgt aus de Gleichung \eqref{eq:Gesamter_strom}
\begin{align}
\frac{\mathrm{d} I_{\mathrm{S}}}{\mathrm{d} U_{\mathrm{S}}}= \frac{\mathrm{d} I_{\mathrm{e}1}}{\mathrm{d} U_{\mathrm{S}}}= -\frac{\mathrm{d} I_{\mathrm{e}2}}{\mathrm{d} U_{\mathrm{S}}}.
\label{eq.Ableitung_strom}
\end{align}
Durch einsetzen der Gleichungen \eqref{eq:ELektronenlaufbereich_Doppel_sonde1} und \eqref{eq:ELektronenlaufbereich_Doppel_sonde2} in die Gleichung \eqref{eq.Ableitung_strom} folgt
\begin{align}
I_{\mathrm{e}1,\mathrm{sat}}\ \mathrm{exp} \left( \frac{e U_1}{T_{\mathrm{e}}} \right) \frac{\mathrm{d} U_1}{\mathrm{d} U_{\mathrm{S}}} + I_{\mathrm{e}2,\mathrm{sat}}\ \mathrm{exp} \left( \frac{e U_2}{T_{\mathrm{e}}} \right)  \left( \frac{\mathrm{d} U_1}{\mathrm{d} U_{\mathrm{S}}} -1 \right) =0.
\end{align}
Bei einer angelegten Spannung von $U=0$ V gilt $U_1=U_2=U_{\mathrm{fl}}$, daraus folgt
\begin{align}
\left. \frac{\mathrm{d} U_1}{\mathrm{d} U_{\mathrm{S}}}\right\vert_{U_{\mathrm{S}}=0}= \frac{I_{\mathrm{e}2,\mathrm{sat}}}{I_{\mathrm{e}1,\mathrm{sat}}+I_{\mathrm{e}2,\mathrm{sat}}}=\frac{I_{\mathrm{i}2,\mathrm{sat}}}{I_{\mathrm{i}1,\mathrm{sat}}+I_{\mathrm{i}2,\mathrm{sat}}}.
\label{eq:Ableitung_2}
\end{align}  
Aus den Gleichungen \eqref{eq.Ableitung_strom} und \eqref{eq:Ableitung_2} kann die Ableitung der Sondenkennlinie bei $U_{\mathrm{S}}=0$ 
\begin{align}
\left. \frac{\mathrm{d} I_{\mathrm{S}}}{\mathrm{d} U_{\mathrm{S}}}\right\vert_{U_{\mathrm{S}}=0} = \frac{e}{T_{\mathrm{e}}}  \frac{I_{\mathrm{i}2,\mathrm{sat}}}{I_{\mathrm{i}1,\mathrm{sat}}+I_{\mathrm{i}2,\mathrm{sat}}}  I_{\mathrm{e}1,\mathrm{sat}}\ \mathrm{exp} \left( \frac{e(\Phi_{\mathrm{p}}-\Phi_{\mathrm{fl}})}{T_{\mathrm{e}}}\right)
\end{align}
bestimmt werden. Gilt für die angelegte Spannung $U_1=U_{\mathrm{fl}}$, folgt damit
\begin{align}
I_{\mathrm{i}1,\mathrm{sat}} =I_{\mathrm{e}1,\mathrm{sat}}\ \mathrm{exp} \left( -\frac{e(\Phi_{\mathrm{p}}-U_\text{fl})}{T_{\mathrm{e}}} \right).
\end{align}
Daraus ergibt sich
\begin{align}
\left. \frac{\mathrm{d} I_{\mathrm{S}}}{\mathrm{d} U_{\mathrm{S}}}\right\vert_{U_{\mathrm{S}}=0} = \frac{e}{T_{\mathrm{e}}}\frac{I_{\mathrm{i}1,\mathrm{sat}} I_{\mathrm{i}2,\mathrm{sat}} }{I_{\mathrm{i}1,\mathrm{sat}}+I_{\mathrm{i}2,\mathrm{sat}}}.
\end{align}
Durch fitten von zwei Geraden an die beiden Ionensättigungsbereiche und kann durch eine Interpolation der Ionensättigungsstrom bestimmt werden. Aus den Schnittpunkten der Geraden mit der Achse für die Stromstärke, können die Ionensättigungsströme ermittelt werden. Daraus kann die Plasmadichte bestimmt werden.  Aus der Steigung der Kennlinie im Elektronenlaufbereich kann die Elektronentemperatur bestimmt werden. 


%\section{Versuchsdurchführung}
\subsection{Aufbau}
Der Hauptbestandteil des Versuchsaufbaus ist eine Glasröhre, in dem das Plasma erzeugt werden kann. In der Glasröhre befinden sich die Elektroden zur Erzeugung des Plasmas mittels einer Glimmentladung, wie im Kapitel \ref{sec:plasma} beschrieben. In der Röhre befinden sich auch die beiden Langmuir-Sonden, die je nach Bedarf als Doppel- oder Einzelsonde benutzt werden können. Die Sonden sind beweglich an der Glasröhre befestigt, um die radiale Position verändern zu können. Der Gasdruck wird durch ein Flussgleichgewicht kontrolliert. An einem Ende der Glasröhre ist die Gaszufuhr angeschlossen und am anderen Ende eine Vakuumpumpe. Bei dem verwendetem Gas handelt es sich um Argon. Durch eine Änderung de Menge des einströmenden Gases kann der Druck eingestellt werden. Zur Druckmessung ist ein Druckmessgerät angeschlossen. Durch den ständigen Fluss des Gases kann die Menge an Verunreinigungen in der Glasröhre reduziert werden. Die Spannung zur Erzeugung des Plasmas und die Spannung die an der Langmuir-Sonde angeschlossen ist, wird durch zwei Spannungsgeräte manuell kontrolliert. Zur Messung der Kennlinie wird ein LabView Programm benutzt. 
\subsection{Durchführung}
Im erstem Versuchsteil soll das Plasma mit einer Langmuir-Sonde untersucht werden. Die Kennlinie der Langmuir-Sonde soll für fünf unterschiedliche Gasdrücke gemessen  werden. Aus den Kennlinien k\"onnen dann Elektronentemperatur, Dichte, Ionisationsgrad, Debye-Länge und Plasmafrequenz bestimmt werden. Aufgetragen werden sollen die Ergebnisse in Diagramme in Abhängigkeit vom gemessenem Druck.
Im zweitem Versuchsteil soll das Plasma mit einer Doppel-Sonde untersucht werden. Dazu wird, analog zum erstem Versuchsteil, für fünf unterschiedliche Drücke die Kennlinie gemessen. Aus der Kennlinie sollen die gleichen Parameter wie aus der Kennlinie der Langmuir-Sonde bestimmt werden. Im drittem Versuchsteil sollen die Radialprofile der Dichte und der Temperatur untersucht werden. Dazu wird mit einer Langmuir-Sonde an unterschiedlichen radialen Positionen die Kennlinie gemessen. Aus den Kennlinien sollen die Elektronentemperatur und die Dichte ermittelt werden. Dieser Versuchsteil wird für mehrere Endladungsströme durchgeführt.
\section{Auswertung}
\subsection{Kalibrierkurve}
Vor der eigentlichen Messung der Hysteresekurven, soll die Kalibrierkurve für das Magnetfeld gemessen werden. Dazu wird mittels einer Hall-Sonde das Magnetfeld gemessen. Die gemessenen Daten sind in der Abbildung \ref{fig:Kalibrierung} dargestellt. Die gemessene Kalibrierkurve ist nicht linear. Die Ursache für die Nichtlinearität ist die Sättigung des Eisenkerns. 
\begin{figure}[H]
\centering
\includegraphics[scale=0.8]{../Messdaten/auswertung/Kalibrierung.pdf}
\caption{ Das Magnetfeld für unterschiedliche Spannungen an der Position der Probe innerhalb der Spule. Zu sehen ist ein nichtlinearer Zusammenhang der Kalibrierkurve. }
\label{fig:Kalibrierung}
\end{figure}


\subsection{Temperaturabhängigkeit der Hysteresekurven}
Für die Berechnung des Kerr-Winkels wird der Zusammenhang zwischen der Drehung der Mikrometerschraube mit der gemessenen Spannung benötigt. Die gemessenen Werte sind in der Tabelle \ref{tab:Mikrometerschraube} dargestellt. Aus der Stellung der Mikrometerschraube kann der Winkel berechnet werden, wobei gilt $ 1 \mathrm{Tick}\ \widehat{=}\ 0.024\ {}^\circ$. Gut erkennbar ist der erwartete lineare Zusammenhang zwischen Kerr-Winkel und Spannung. Ohne das ein Fit notwendig ist, kann aufgrund der Daten aus der Tabelle \ref{tab:Mikrometerschraube} eine Geradengleichung bestimmt werden. Es gilt für alle gemessenen Spannungen das Verhältnis
\begin{equation}
1 \mathrm{Tick}\ \widehat{=}\ 0.01\ \mathrm{V}.
\end{equation}
Unter Berücksichtigung des Umrechnungsfaktors zwischen der Stellung der Mikrometerschraube und Winkel $\phi$ lautet die Geradengleichung 
\begin{equation}
\phi = 2.4\ \frac{{ }^\circ}{\mathrm{V}} \cdot U,
\label{eq:Kerr}
\end{equation}
wobei $U$ die Spannung ist. 

\begin{table}[h]
    \centering
    \caption{
        Zusammenhang zwischen Stellung der Mikrometerschraube des Polarisationsfilters und  des Kerr-Winkels mit der Spannung.
        }
    \label{tab:Mikrometerschraube}
    \begin{tabular}{r|c|l}
    Mikrometerschraube [ticks] & Kerr-Winkel [${}^\circ$]& Spannung [V] \\\hline
    0  & 0 & 0
 \\
    10 & 0.24 & 0.1
 \\
    20 & 0.48 & 0.2
 \\
    30 & 0.72 & 0.3
 \\
    40 & 0.96 & 0.4
 \\
    50 & 1.2 & 0.5 \\   
    \end{tabular}
\end{table}
Die unbearbeitete Hysteresekurve in der Abbildung \ref{fig:hysterese_original} ist schräg und nicht gerade. 
Die Ursache dafür ist, dass in der Polymerschicht die sich auf der magnetooptischen Disk befindet es auch zu einer Polarisationsänderung kommt. 
Diese Polarisationsänderung gibt es zusätzlich zu der Polarisationsänderung durch den Kerr-Effekt. 
Dieser Einfluss der Polymerschicht wurde aus den Hysterekurven rausgerechnet. 
Die korrigierten Hysteresekurven für unterschiedliche Temperaturen  sind in den Abbildungen \ref{fig:hysterese_temp_1} und \ref{fig:hysterese_temp_2} dargestellt. 


\begin{figure}[H]
\centering
\includegraphics[scale=0.8]{../Messdaten/auswertung/hysterese_7_original.pdf}
\caption{ Unbearbeitete Hysteresekurve bei einer Temperatur von $31,4\ {}^\circ$C.}
\label{fig:hysterese_original}
\end{figure}


\begin{figure}[H]
\centering
    \subfig{../Messdaten/auswertung/hysterese_7.pdf}
    \subfig{../Messdaten/auswertung/hysterese_8.pdf}
    \subfig{../Messdaten/auswertung/hysterese_9.pdf}
    \subfig{../Messdaten/auswertung/hysterese_10.pdf}
    \subfig{../Messdaten/auswertung/hysterese_11.pdf}
    \subfig{../Messdaten/auswertung/hysterese_12.pdf}
    
    
    
\caption{ Hysteresekurven für unterschiedliche Temperaturen.  }
\label{fig:hysterese_temp_1}
\end{figure}

\begin{figure}[H]
\centering
\subfig{../Messdaten/auswertung/hysterese_13.pdf}
    \subfig{../Messdaten/auswertung/hysterese_14.pdf}
    \subfig{../Messdaten/auswertung/hysterese_15.pdf}
    \subfig{../Messdaten/auswertung/hysterese_16.pdf}
    \subfig{../Messdaten/auswertung/hysterese_17.pdf}
\caption{ Hysteresekurven für unterschiedliche Temperaturen. }
\label{fig:hysterese_temp_2}
\end{figure}

Aus den Hysteresekurven kann der Kerr-Winkel für die Sättigung der Magnetisierung berechnet werden.
Dazu wird aus den Hysteresekurven die Differenz zwischen minimaler und maximaler Spannung abgelesen.
Die Häfte dieser Differenz entspricht dem Kerr-Winkel.
Die Spannung kann mittels der Gleichung \eqref{eq:Kerr} in den Winkel umgerechnet werden.
Die berechneten Kerr-Winkel für unterschiedliche Temperaturen sind in der Abbildung \ref{fig:Kerr_Winkel} dargestellt.
Wie zu erkennen ist, nimmt der Kerr-Winkel f"ur steigende Temperaturen leicht ab, wie es aus Abbildung \vref{fig:temperatur} erwartet wird.
\begin{figure}[H]
\centering
\includegraphics[scale=0.8]{../Messdaten/auswertung/Kerr_Winkel.pdf}
\caption{Der Kerr-Winkel für die Sättigung der Magnetisierung für unterschiedliche Temperaturen ermimttelt aus den Hysteresekurven.}
\label{fig:Kerr_Winkel}
\end{figure}


\begin{figure}[H]
\centering
\includegraphics[scale=0.8]{../Messdaten/auswertung/Kompensationstemperatur.pdf}
\caption{Die aus den Hysteresekurven ermittelten Koerzitivfeldstärken für unterschiedliche Temperaturen 
und der Fit mit der Gleichung \eqref{eq:koerzitivfeldstaerke}.}
\label{fig:koerzitivfeldstaerke}
\end{figure}

Die Koerzitivfeldstärke $H_\mathrm{c}$, also die magnetische Feldstärke die für die vollständige Entmagnetisierung notwendig ist, kann den Hysterekurven entnommen werden. 
Die Koerzitivfeldstärke entspricht der Hälfte der Breite der Hysterekurven. 
Die Koerzitivfeldstärken in Abhängigkeit der Temperatur sind in der Abbildung \ref{fig:koerzitivfeldstaerke} dargestellt. 
Aus den Koerzitivfeldstärken kann die Kompensationstemperatur $T_\mathrm{comp}$ bestimmt  
werden.
Dazu muss die Funktion  
\begin{equation}
H_\mathrm{c} = \frac{H_0 \cdot T}{T-T_\mathrm{comp}}
\label{eq:koerzitivfeldstaerke} 
\end{equation}
als Fitfunktion benutzt werden.
Die Fitparameter sind
\begin{align}
H_0 & = 59.70 \pm 4.78\ \mathrm{mT}\\
T_\mathrm{comp} & = 279.45 \pm 2.73\ \mathrm{K} .
\label{eq:fitwerte}
\end{align}

 

%\FloatBarrier
%\section{Unsicherheitsbetrachtung}
Der gr\"o\ss te Unsicherheitsfaktor in der Messung ist das veraltete Barometer.
Es zeigt Abweichungen von Teils sogar mehr als dem angegebenen Fehler $\Delta p=\sidp$.
Dies macht es quasi unm\"oglich einen pr\"azisen Druck einzustellen.
Die Messung der Str\"ome und Spannungen hingegen ist ziemlich genau.
Aufgrund der Unsicherheit der Messger\"ate sowie angenommener statistischer Fehler werden hier Fehler von $\Delta I=\sidI$ und $\Delta U=\sidU$ angenommen.
Dar\"uber hinaus k\"onnen die Fehler f\"ur Fitparameter direkt aus der Diagonalen ihrer Kovarianzmatrix berechnet werden.
F\"ur die Fits wird die Methode der kleinsten Quadrate unter Ber\"ucksichtigung des Fehlers von $I$ verwendet.

Der Fehler abgeleiteter Gr\"o\ss en wird mittels Fehlerfortpflanzung bestimmt.
Wenn $X$ von $U$ und $I$ abh\"angt dann ist der Fehler von $X$
\begin{equation}
\Delta X = \abs{\frac{\partial X}{\partial U}}\Delta U +  \abs{\frac{\partial X}{\partial I}}\Delta I
\label{eq:err}
\end{equation}
Weitere Fehlerquellen sind Temperatur- und Druckschwankungen im Plasma, sowie die M\"oglichkeit, dass der Elektronenstrom in die S\"attigung \"ubergeht und das Plasma um sich herum ver\"andert.
Letzteres ist an einem ist an einem helleren Gl\"uhen um die Sonde herum zu erkennen und kann daher vermieden werden.

\section{Zusammenfassung}
Im Versuch werden die Kennlinien einer Langmuir- und einer Doppel-Sonde im Argon-Plasma untersucht.
Dabei wird die Langmuir-Sonde mit einer negativen Spannung betrieben.
Dies f\"uhrt dazu, dass die Sonde negativ geladen ist und Elektronen im Plasma abst\"o\ss t.
Die Kennlinie zeigt dabei eine S\"attigung bei stark negativen Spannungen.
Mit steigender Spannung ergibt sich zun\"achst ein exponentieller Anstieg des Stroms, der bald in eine lineare Steigung \"ubergeht.
F\"ur die Doppelsonde verl\"auft die Kurve im Elektronenanlaufbereich linear.
In den Elektronenstroms\"attigungsbereichen verl\"auft die Kurve ebenfalls linear, allerdings mit einer geringeren Steigung.
Der lineare Anteil in der S\"attigung kommt dabei von der zylindrischen Form des Sondenkopfes.

F\"ur beide Sonden werden die Elektronentemperatur $T_e$, die Elektronendichte $n_e$, der Ionisationsgrad $\alpha$, die Debye-L\"ange $\lambda_\text{D}$, sowie die Plasmafrequenzen f\"ur Elektronen $\omega_e$ und f\"ur Ionen $\omega_i$ untersucht.
Es zeigen sich dabei keine klaren Trends in Abh\"angigkeit des Gasdrucks $p$, was haupts\"achlich dem ungenauen Barometer geschuldet ist.
Allerdings stellt sich heraus, dass $n_e$, $\omega_e$ und $\omega_i$ mit steigendem $p$ steigen, wh\"ahrend die anderen Werte fallen.
Elektronen Schwingen aufgrund ihrer geringeren Masse schneller.
Darum ist $\omega_e$ um etwa drei Gr\"o\ss enordnungen h\"oher als $\omega_i$.
%Der Ionisationsgrad sinkt leicht, was darauf schlie\ss en l\"asst, dass haupts\"achlich unselbstst\"andige Entladung vorliegt, die durch die Anwesenheit von mehr dichteren Atomen nicht beg\"unstigt wird.
Der abnehmende Ionisationsgrad kann darauf zur\"uckgef\"uhrt werden, dass die Durchschlagsspannung mit steigendem Druck steigt (Paschen-Beziehung).
Die Anwesenheit von mehr potentiell ionisierbaren Atomen verleitet dennoch zu einer h\"oheren Anzahl an freien Elektronen und Ionen.
Diese wiederum schirmen elektrische Felder st\"arker ab, was die Debye-L\"ange verkleinert.
Die Plasmafrequenzen steigen ebenfalls aufgrund der h\"oheren Ladungstr\"agerdichte.
Die Elektronentemperatur hingegen sinkt.
H\"aufigere St\"o\ss e f\"uhren zu geringeren mittleren Wegstrecken, auf denen ein Elektron beschl\"aunigt werden kann, bevor es St\"o\ss t.

Au\ss erdem wird die Abh\"angigkeit der genannten Gr\"o\ss en von der radialen Position der Sonde im Plasma untersucht.
Dabei zeigt sich leider kein Trend, was allerdings wohl darauf zur\"uckzuf\"uhren ist, dass die mit der Messung abgedeckten Bereiche zu klein sind.

Zuletzt wird die Abh\"angigkeit vom Entladungsstrom $I_\text{E}$ untersucht.
Es zeigt sich, dass $n_e$, $\alpha$, $\omega_e$ und $\omega_i$ mit steigendem $I_\text{E}$ steigen; die anderen Werte fallen.
Der h\"ohere Strom (bei h\"oherer Spannung) f\"uhrt zu mehr Ionisation, was Ionisationsgrad und Elektronendichte erh\"oht.
Dies wiederum verst\"arkt die elektrische Abschirmung und senkt die Debye-L\"ange, w\"ahrend die Plasmafrequenzen steigen.
Wieder sinkt die Elektronentemperatur aufgrund h\"aufigerer St\"o\ss e.
%\printbibliography
%\FloatBarrier
\bibliography{mybib}{}
\end{document}
