% Alex header 
\documentclass[aps,amsmath,amssymb,amsfont]{revtex4-1}
\usepackage[utf8]{inputenc}
\usepackage[T1]{fontenc}
\usepackage[german]{babel}
\usepackage{float}
\usepackage{graphicx}
%\usepackage{siunitx}
\usepackage[colorlinks=true,linkcolor=black]{hyperref}
\usepackage{lmodern}
\usepackage{physics}
\usepackage{mathpazo}
\usepackage{here}
\usepackage{amsmath}
\usepackage{mathtools}
\usepackage[utf8]{inputenc}
\usepackage[T1]{fontenc}
\usepackage{lmodern}
\usepackage{subcaption}
\usepackage{placeins}
\usepackage{braket}
\numberwithin{equation}{section}
\usepackage{tikz}
\usetikzlibrary{arrows, calc}
\usepackage[europeanresistors,americaninductors]{circuitikz}
\makeatletter
\AtBeginDocument{
\def\tocname{Inhalt}
	\def\andname{and}
	\def\Dated@name{Dated: }
	\def\figurename{Abb.}
	\def\tablename{Tab.}
}
\makeatother


\newcommand{\subfig}[1]{\begin{subfigure}[t]{0.49\textwidth}
        \centering
        \includegraphics[width=1.1\textwidth]{#1}
        \caption{}	
    \end{subfigure}}
\newcommand{\pictures}[1]{\begin{subfigure}[t]{0.49\textwidth}
        \centering
        \includegraphics[width=1.1\textwidth]{#1}
        \caption{}
    \end{subfigure}}


%Michael header
%\documentclass[
%    captions=nooneline
%    ,headinclude
%    ,headsepline
%    ,bibliography=totocnumbered
%]{scrartcl}
%\usepackage[twoside,paper=a4paper,left=35mm,right=20mm,top=30mm,bottom=35mm]{geometry} 

%\usepackage[T1]{fontenc}
%\usepackage[utf8]{inputenc}
\usepackage[german]{varioref}

%%% GRAPHIKEN %%%
%\usepackage{graphicx}
\usepackage[labelfont=bf,justification=justified,format=plain]{caption}
\usepackage{subcaption}
\usepackage{placeins}
\usepackage{wrapfig}
%\usepackage{floatrow}
%% DREHUNG VON GRAFIKEN
%\usepackage{rotating}
%\usepackage{pdflscape}

%%% Tabellen %%%
%\usepackage{threeparttable}
%\usepackage{booktabs}
%\usepackage{multirow}

%%% MATHE %%%
%\usepackage{mathtools}
%\usepackage{physics}
%\usepackage{upgreek}
%\usepackage{amssymb}
\usepackage[separate-uncertainty=true,
            list-units = single,
            list-separator = {;\,},
            exponent-product=\cdot,%
            ]{siunitx}
%\DeclareSIUnit\molar{\mole\per\l}

%%% ZEICHENUMGEBUNG %%%
\usepackage{pgf}


%%% BIBLIOGRAPHIE %%%
%\usepackage[style=numeric-comp,backend=biber,sortlocale=de_DE,natbib=true]{biblatex}
%\addbibresource{Quellen.bib}
%\usepackage[babel,german]{csquotes}

%%% EIGENE DEFINITIONEN %%%
%\newcommand{\iu}{{i\mkern1mu}}
%\usepackage{txfonts} % |R, |C

%\usepackage{hyperref}

%%% INCLUDE VALUES %%%
%\newcommand{\bigtableplus}{
\begin{tabular}{S[table-format=+3.1]S[table-format=+2]|S[table-format=3]S[table-format=3]S[table-format=3]S[table-format=3]S[table-format=3]}
{$\alpha$ [\si{\degree}]}&{$\beta$ [\si{\degree}]}&\multicolumn{5}{c}{$C(\alpha,\beta)$ [counts/s]}\\\hline
22.5&0&170&172&166&165&161\\
22.5&90&624&627&616&622&606\\
112.5&0&644&648&632&639&627\\
112.5&90&176&177&179&178&173\\\hline
22.5&-45&158&157&145&151&154\\
22.5&45&628&632&656&641&631\\
112.5&-45&646&665&661&652&680\\
112.5&45&131&149&145&150&151\\\hline
-22.5&0&145&141&141&153&141\\
-22.5&90&599&593&592&577&593\\
67.5&0&640&623&637&623&634\\
67.5&90&123&121&120&127&121\\\hline
-22.5&-45&636&642&642&642&649\\
-22.5&45&166&171&168&174&183\\
67.5&-45&196&210&201&211&203\\
67.5&45&558&563&562&562&549\\\hline
\end{tabular}}

\newcommand{\bigtableminus}{
\begin{tabular}{S[table-format=+3.1]S[table-format=+2]|S[table-format=3]S[table-format=3]S[table-format=3]S[table-format=3]S[table-format=3]}
{$\alpha$ [\si{\degree}]}&{$\beta$ [\si{\degree}]}&\multicolumn{5}{c}{$C(\alpha,\beta)$ [counts/s]}\\\hline
22.5&0&84&91&89&84&88\\
22.5&90&634&629&630&624&615\\
112.5&0&625&618&622&624&625\\
112.5&90&79&73&76&80&75\\\hline
22.5&-45&441&448&453&450&448\\
22.5&45&267&265&260&264&272\\
112.5&-45&274&269&280&264&268\\
112.5&45&407&423&415&421&430\\\hline
-22.5&0&248&258&252&254&247\\
-22.5&90&448&459&441&450&444\\
67.5&0&535&514&502&503&514\\
67.5&90&218&211&213&215&219\\\hline
-22.5&-45&94&97&96&102&96\\
-22.5&45&616&627&607&630&611\\
67.5&-45&647&637&632&621&631\\
67.5&45&108&103&107&109&111\\\hline
\end{tabular}}

\newcommand{\bigtabletom}{
\begin{tabular}{S[table-format=2]|ll|S[table-format=+2]S[table-format=+2]|S[table-format=2]S[table-format=2]||S[table-format=3]S[table-format=3]S[table-format=3]S[table-format=3]S[table-format=3]}
&{S1}&{S2}&{P1 [\si{\degree}]}&{L1 [\si{\degree}]}&{P2 [\si{\degree}]}&{L2 [\si{\degree}]}&\multicolumn{5}{c}{$C(\alpha,\beta)$ [counts/s]}\\\hline
1&$\state{H}$&$\state{H}$&0&0&0&0&49&48&52&48&49\\
2&$\state{H}$&$\state{V}$&0&0&90&0&647&647&647&641&647\\
3&$\state{V}$&$\state{V}$&90&0&90&0&49&47&43&45&47\\
4&$\state{V}$&$\state{H}$&90&0&0&0&683&669&660&665&686\\
5&$\state{R}$&$\state{H}$&90&45&0&0&413&396&402&409&409\\
6&$\state{R}$&$\state{V}$&90&40&90&0&300&283&297&292&293\\
7&$\state{D}$&$\state{V}$&45&45&90&0&419&434&426&420&416\\
8&$\state{D}$&$\state{H}$&45&45&0&0&303&307&312&308&303\\
9&$\state{D}$&$\state{R}$&45&45&45&0&399&391&388&392&378\\
10&$\state{D}$&$\state{D}$&45&45&45&45&621&622&628&628&618\\
11&$\state{R}$&$\state{D}$&45&0&45&45&382&378&385&378&392\\
12&$\state{H}$&$\state{D}$&0&0&45&45&259&267&268&269&270\\
13&$\state{V}$&$\state{D}$&90&0&45&45&436&422&435&425&428\\
14&$\state{V}$&$\state{L}$&90&0&0&45&351&342&345&338&344\\
15&$\state{H}$&$\state{L}$&0&0&0&45&394&385&401&387&396\\
16&$\state{R}$&$\state{L}$&0&-45&0&45&55&54&50&55&48\\
\hline
\end{tabular}}

\newcommand{\Ma}{
\frac{1}{2}\begin{pmatrix}
2&1&-1+i&-1-i\\
1&0&0&0\\
-1-i&0&0&i\\
-1+i&0&-i&0
\end{pmatrix}}

\newcommand{\Mb}{
\frac{1}{2}\begin{pmatrix}
0&1&-1+i&0\\
1&0&-1+i&0\\
-1-i&-1-i&2&i\\
0&0&-i&0
\end{pmatrix}}

\newcommand{\Mc}{
\frac{1}{2}\begin{pmatrix}
0&1&0&0\\
1&2&-1+i&-1-i\\
0&-1-i&0&i\\
0&-1+i&-i&0
\end{pmatrix}}

\newcommand{\Md}{
\frac{1}{2}\begin{pmatrix}
0&1&0&-1-i\\
1&0&0&-1-i\\
0&0&0&i\\
-1+i&-1+i&-i&2
\end{pmatrix}}

\newcommand{\Me}{
\frac{1}{2}\begin{pmatrix}
0&-1-i&0&2i\\
-1+i&0&0&0\\
0&0&0&-1-i\\
-2i&0&1+i&0
\end{pmatrix}}

\newcommand{\Mf}{
\frac{1}{2}\begin{pmatrix}
0&-1-i&0&0\\
-1+i&0&-2i&0\\
0&2i&0&1-i\\
0&0&1+i&0
\end{pmatrix}}

\newcommand{\Mg}{
\frac{1}{2}\begin{pmatrix}
0&-1-i&0&0\\
-1+i&0&2&0\\
0&2&0&-1+i\\
0&0&-1-i&0
\end{pmatrix}}

\newcommand{\Mh}{
\frac{1}{2}\begin{pmatrix}
0&-1-i&0&2\\
-1+i&0&0&0\\
0&0&0&-1+i\\
2&0&-1-i&0
\end{pmatrix}}

\newcommand{\Mi}{
\frac{1}{2}\begin{pmatrix}
0&2i&0&0\\
-2i&0&0&0\\
0&0&0&-2i\\
0&0&2i&0
\end{pmatrix}}

\newcommand{\Mj}{
\frac{1}{2}\begin{pmatrix}
0&2&0&0\\
2&0&0&0\\
0&0&0&2\\
0&0&2&0
\end{pmatrix}}

\newcommand{\Mk}{
\frac{1}{2}\begin{pmatrix}
0&2i&0&0\\
-2i&0&0&0\\
0&0&0&2i\\
0&0&-2i&0
\end{pmatrix}}

\newcommand{\Ml}{
\frac{1}{2}\begin{pmatrix}
0&-1-i&2&0\\
-1+i&0&0&0\\
2&0&0&-1-i\\
0&0&-1+i&0
\end{pmatrix}}

\newcommand{\Mm}{
\frac{1}{2}\begin{pmatrix}
0&-1-i&0&0\\
-1+i&0&0&2\\
0&0&0&-1-i\\
0&2&-1+i&0
\end{pmatrix}}

\newcommand{\Mn}{
\frac{1}{2}\begin{pmatrix}
0&-1+i&0&0\\
-1-i&0&0&2i\\
0&0&0&1-i\\
0&-2i&1+i&0
\end{pmatrix}}

\newcommand{\Mo}{
\frac{1}{2}\begin{pmatrix}
0&-1+i&-2i&0\\
-1-i&0&0&0\\
2i&0&0&1-i\\
0&0&1+i&0
\end{pmatrix}}

\newcommand{\Mp}{
\frac{1}{2}\begin{pmatrix}
0&2&0&0\\
2&0&0&0\\
0&0&0&-2\\
0&0&-2&0
\end{pmatrix}}

\newcommand{\rhonormal}{
\begin{pmatrix}
0.035+0.000i &-0.034+0.056i &-0.057-0.032i &-0.038+0.032i\\
-0.034-0.056i &0.033+0.000i &0.054+0.037i &0.049-0.011i\\
-0.057+0.032i &0.054-0.037i &0.457+0.000i &0.120-0.000i\\
-0.038-0.032i &0.049+0.011i &0.407+0.000i &0.476+0.000i
\end{pmatrix}}

\newcommand{\drhonormal}{
\begin{pmatrix}
0.001+0.000i &0.025+0.024i &0.005+0.006i &0.007+0.009i\\
0.025+0.024i &0.002+0.000i &0.007+0.006i &0.009+0.007i\\
0.005+0.006i &0.007+0.006i &0.007+0.000i &0.021+0.029i\\
0.007+0.009i &0.009+0.007i &0.024+0.029i &0.013+0.000i
\end{pmatrix}}

\newcommand{\evala}{0.706}\newcommand{\eveca}{
\begin{pmatrix}
-0.098+0.028i\\
0.110+0.018i\\
0.467-0.022i\\
0.871+0.000i
\end{pmatrix}}

\newcommand{\evalb}{0.259}\newcommand{\evecb}{
\begin{pmatrix}
0.027+0.209i\\
0.092-0.145i\\
-0.472+0.073i\\
0.835+0.000i
\end{pmatrix}}

\newcommand{\evalc}{-0.036}\newcommand{\evecc}{
\begin{pmatrix}
0.719+0.000i\\
0.264+0.627i\\
-0.005-0.112i\\
0.046+0.068i
\end{pmatrix}}

\newcommand{\evald}{0.071}\newcommand{\evecd}{
\begin{pmatrix}
0.687+0.000i\\
-0.195-0.606i\\
0.237-0.058i\\
-0.160+0.195i
\end{pmatrix}}

\newcommand{\rhobell}{
\begin{pmatrix}
0.000+0.000i &0.001-0.056i &0.004+0.013i &-0.007-0.008i\\
0.001+0.056i &0.067+0.000i &-0.100-0.013i &-0.012-0.056i\\
0.004-0.013i &-0.100+0.013i &0.730+0.000i &0.134+0.000i\\
-0.007+0.008i &-0.012+0.056i &-0.153-0.000i &0.202+0.000i
\end{pmatrix}}

\newcommand{\drhobell}{
\begin{pmatrix}
0.027+0.024i &0.027+0.024i &0.014+0.014i &0.014+0.014i\\
0.027+0.024i &0.027+0.024i &0.014+0.014i &0.014+0.014i\\
0.014+0.014i &0.014+0.014i &0.032+0.029i &0.032+0.029i\\
0.014+0.014i &0.014+0.014i &0.032+0.029i &0.032+0.029i
\end{pmatrix}}



%\includeonly{}

\tikzset{
    declare function={% in case of CVS which switches the arguments of atan2
        atan3(\a,\b)=ifthenelse(atan2(0,1)==90, atan2(\a,\b), atan2(\b,\a));},
    kinky cross radius/.initial=+.125cm,
    @kinky cross/.initial=+, kinky crosses/.is choice,
    kinky crosses/left/.style={@kinky cross=-},kinky crosses/right/.style={@kinky cross=+},
    kinky cross/.style args={(#1)--(#2)}{
        to path={
            let \p{@kc@}=($(\tikztotarget)-(\tikztostart)$),
            \n{@kc@}={atan3(\p{@kc@})+180} in
            -- ($(intersection of \tikztostart--{\tikztotarget} and #1--#2)!%
            \pgfkeysvalueof{/tikz/kinky cross radius}!(\tikztostart)$)
            arc [ radius     =\pgfkeysvalueof{/tikz/kinky cross radius},
            start angle=\n{@kc@},
            delta angle=\pgfkeysvalueof{/tikz/@kinky cross}180 ]
            -- (\tikztotarget)}}}

\begin{document}
	\begin{titlepage}
	\centering
	\par\vspace{1cm}
	{\scshape\LARGE Universität Stuttgart \par}
	\vspace{1cm}
	{\scshape\Large  Fortgeschrittenen Praktikum \\ Wintersemester 2018/2019\par}
	\vspace{1.5cm}
	{\huge\bfseries Ionenschallwellen im Plasma\par}
	\vspace{2cm}
	{\Large\itshape Gruppe: M09\\ Alexander Sattler  \\  Michael Marquardt    \\ Versuchsdatum: Montag 17. Dezember 2018\\ Betreuer: Alf Köhn\\ \par}
	
	\end{titlepage}


\tableofcontents

%\siUerr
%\siIerr
\section{Theory}
\subsection{Spontaneous parametric downconversion} 
Spontaneous parametric downconversion is the spontaneous splitting of a photon in two different photons. That can be done by shining with a laser beam on a crystal, which has nonlinear effects. Under certain circumstances the both emitted photons of the crystal are entangled. 
As for every process momentum and energy conservation must be fulfilled. Because the energy for a photon is $E = \hbar \omega$, one can consider the frequencies to satisfiy the energy conservation. The pumping laser from the laser has the frequency $\omega_{\mathrm{p}}$. The resulting photons, which are named as signal and idler photon, after the splitting must have frequencies below $\omega_{\mathrm{p}}$. The energy conservation can be written as
\begin{equation}
    \omega_{\mathrm{p}} = \omega_{\mathrm{i}} + \omega_{\mathrm{s}},
\end{equation}
where $\omega_{\mathrm{i}}$ is the frequency of the idler photon and $\omega_{\mathrm{s}}$ of the signal photon. Here we are interested in the degenerated case $ \omega_{\mathrm{s}} = \omega_{\mathrm{i}}$ to make the photons indistinguishable by frequency. 
As the momentum can be calculated by $\vec{p} = \hbar \vec{k}$, for  the momentum conversation one con consider the $\vec{k}$ vectors of the tree photons. One obtains
\begin{equation}
    \vec{k}_{\mathrm{p}} =  \vec{k}_{\mathrm{i}}  + \vec{k}_{\mathrm{s}},
    \label{eq:phase_matching}
\end{equation}
what is called phase matching. 
The explained process is not possible for free electrons, but in a nonlinear medium. The nonlinearity refers to the polarization which is not linear dependent on the electric field
\begin{equation}
    \vec{P}_{\mathrm{i}} = \varepsilon_0 \left( \chi^{(1)} \vec{E}_{\mathrm{j}} 
    + \chi^{(2)} \vec{E}_1 \vec{E}_2 
    + \chi^{(3)} \vec{E}_1 \vec{E}_2 \vec{E}_3 + ... \right).
\end{equation}
The understand phase matching it is important to understand how photons propagate in a birefringence media. 

%Talk:
%   - both walk off effect
%   - Phase matching
%   - 2 kegel, deren schnittpunkte
%   - ordinary, extraordinary beam
\subsection{CHSH inequality}
%TALK about:
%   - bell states
%   - entanglement
%   - hanbury brown twiss effect
%   - inequality
%   - how to violate inequality

%\section{Versuchsdurchführung}
\subsection{Aufbau}
Der Hauptbestandteil des Versuchsaufbaus ist eine Glasröhre, in dem das Plasma erzeugt werden kann. In der Glasröhre befinden sich die Elektroden zur Erzeugung des Plasmas mittels einer Glimmentladung, wie im Kapitel \ref{sec:plasma} beschrieben. In der Röhre befinden sich auch die beiden Langmuir-Sonden, die je nach Bedarf als Doppel- oder Einzelsonde benutzt werden können. Die Sonden sind beweglich an der Glasröhre befestigt, um die radiale Position verändern zu können. Der Gasdruck wird durch ein Flussgleichgewicht kontrolliert. An einem Ende der Glasröhre ist die Gaszufuhr angeschlossen und am anderen Ende eine Vakuumpumpe. Bei dem verwendetem Gas handelt es sich um Argon. Durch eine Änderung de Menge des einströmenden Gases kann der Druck eingestellt werden. Zur Druckmessung ist ein Druckmessgerät angeschlossen. Durch den ständigen Fluss des Gases kann die Menge an Verunreinigungen in der Glasröhre reduziert werden. Die Spannung zur Erzeugung des Plasmas und die Spannung die an der Langmuir-Sonde angeschlossen ist, wird durch zwei Spannungsgeräte manuell kontrolliert. Zur Messung der Kennlinie wird ein LabView Programm benutzt. 
\subsection{Durchführung}
Im erstem Versuchsteil soll das Plasma mit einer Langmuir-Sonde untersucht werden. Die Kennlinie der Langmuir-Sonde soll für fünf unterschiedliche Gasdrücke gemessen  werden. Aus den Kennlinien k\"onnen dann Elektronentemperatur, Dichte, Ionisationsgrad, Debye-Länge und Plasmafrequenz bestimmt werden. Aufgetragen werden sollen die Ergebnisse in Diagramme in Abhängigkeit vom gemessenem Druck.
Im zweitem Versuchsteil soll das Plasma mit einer Doppel-Sonde untersucht werden. Dazu wird, analog zum erstem Versuchsteil, für fünf unterschiedliche Drücke die Kennlinie gemessen. Aus der Kennlinie sollen die gleichen Parameter wie aus der Kennlinie der Langmuir-Sonde bestimmt werden. Im drittem Versuchsteil sollen die Radialprofile der Dichte und der Temperatur untersucht werden. Dazu wird mit einer Langmuir-Sonde an unterschiedlichen radialen Positionen die Kennlinie gemessen. Aus den Kennlinien sollen die Elektronentemperatur und die Dichte ermittelt werden. Dieser Versuchsteil wird für mehrere Endladungsströme durchgeführt.
\section{Analysis}


\section{Zusammenfassung}
In diesem Versuch sind die magnetischen Eigenschaften einer magnetooptischen Disk untersucht worden. Gemessen wurden Hysteresekurven für unterschiedliche Temperaturen. Die gemessene Kurven wurde um ihren Offset und den Einfluss der schützenden Polymerschicht korrigiert. Aus den gemessenen Daten ist die Koerzitivfeldstärke ermittelt worden, mit der die Kompensationstemperatur 
\begin{equation}
T_\mathrm{comp}  = 279.45 \pm 2.73\ \mathrm{K} 
\end{equation}
berechnet werden konnte.
Der Kerr-Winkel nimmt bei steigenden Temperaturen leicht ab.
Er bleibt dabei in einem Bereich von \SI{0.7}{\degree} bis \SI{0.95}{\degree}.

%\FloatBarrier
%\section{Unsicherheitsbetrachtung}
Der gr\"o\ss te Unsicherheitsfaktor in der Messung ist das veraltete Barometer.
Es zeigt Abweichungen von Teils sogar mehr als dem angegebenen Fehler $\Delta p=\sidp$.
Dies macht es quasi unm\"oglich einen pr\"azisen Druck einzustellen.
Die Messung der Str\"ome und Spannungen hingegen ist ziemlich genau.
Aufgrund der Unsicherheit der Messger\"ate sowie angenommener statistischer Fehler werden hier Fehler von $\Delta I=\sidI$ und $\Delta U=\sidU$ angenommen.
Dar\"uber hinaus k\"onnen die Fehler f\"ur Fitparameter direkt aus der Diagonalen ihrer Kovarianzmatrix berechnet werden.
F\"ur die Fits wird die Methode der kleinsten Quadrate unter Ber\"ucksichtigung des Fehlers von $I$ verwendet.

Der Fehler abgeleiteter Gr\"o\ss en wird mittels Fehlerfortpflanzung bestimmt.
Wenn $X$ von $U$ und $I$ abh\"angt dann ist der fehler von $X$
\begin{equation}
\Delta X = \abs{\frac{\partial X}{\partial U}}\Delta U +  \abs{\frac{\partial X}{\partial I}}\Delta I
\end{equation}
Weitere Fehlerquellen sind Temperaturschwankungen im Plasma und die M\"oglichkeit, dass der Elektronenstrom in die S\"attigung \"ubergeht und das Plasma um sich herum ver\"andert.
Letzteres ist an einem st an einem helleren Gl\"uhen um die Sonde herum zu erkennen und kann daher vermieden werden.

\section{Zusammenfassung}
Im Versuch werden die Kennlinien einer Langmuir- und einer Doppel-Sonde im Argon-Plasma untersucht.
Dabei wird die Langmuir-Sonde mit einer negativen Spannung betrieben.
Dies f\"uhrt dazu, dass die Sonde negativ geladen ist und Elektronen im Plasma abst\"o\ss t.
Die Kennlinie zeigt dabei zun\"achst eine S\"attigung bei stark negativen Spannungen.
Mit steigender Spannung ergibt sich zun\"achst ein exponentieller Anstieg des Stroms, der bald in eine lineare Steigung \"ubergeht.
F\"ur die Doppelsonde verl\"auft die Kurve im Elektronenanlaufbereich linear.
In den Elektronenstroms\"attigungsbereichen verl\"auft die Kurve ebenfalls linear, allerdings mit einer geringeren Steigung.
Der lineare Anteil in der S\"attigung kommt dabei von der zylindrischen Form des Sondenkopfes.

F\"ur beide Sonden werden die Elektronentemperatur $T_e$, die Elektronendichte $n_e$, der Ionisationsgrad $\alpha$, die Debye-L\"ange $\lambda_\text{D}$, sowie die Plasmafrequenzen f\"ur Elektronen $\omega_e$ und f\"ur Ionen $\omega_i$ untersucht.
Es zeigen sich dabei keine klaren Trends in Abh\"angigkeit des Gasdrucks $p$, was haupts\"achlich dem Ungenauen Barometer geschuldet ist.
Allerdings stellt sich heraus, dass $n_e$, $\omega_e$ und $\omega_i$ mit steigendem $p$ steigen, wh\"ahrend die anderen Werte fallen.
Elektronen Schwingen aufgrund ihrer geringeren Masse schneller.
Darum ist $\omega_e$ um etwa drei Gr\"o\ss enordnungen h\"oher als $\omega_i$.
%Der Ionisationsgrad sinkt leicht, was darauf schlie\ss en l\"asst, dass haupts\"achlich unselbstst\"andige Entladung vorliegt, die durch die Anwesenheit von mehr dichteren Atomen nicht beg\"unstigt wird.
Der abnehmende Ionisationsgrad kann darauf zur\"uckgef\'uhrt werden, dass die Durchschlagsspannung mit steigendem Druck steigt (Paschen-Beziehung).
Die Anwesenheit von mehr potentiell ionisierbaren Atomen verleitet dennoch zu einer h\"oheren Anzahl an freien Elektronen und Ionen.
Diese wiederum schirmen Elektrische Felder st\"arker ab, was die Debye-L\"ange verkleinert.
Die Plasmafrequenzen steigen ebenfalls aufgrund der h\"oheren Ladungstr\"agerdichte.
Die Elektronentemperatur hingegen sinkt.
H\"aufigere St\"o\ss e f\"uhren zu geringeren mittleren Wegstrecken, auf denen ein Elektron beschl\"aunigt werden kann, bevor es St\"o\ss t.

Au\ss erdem wird die Abh\"angigkeit der genannten Gr\"o\ss en von der radialen Position der Sonde im Plasma untersucht.
Dabei zeigt sich leider kein Trend, was allerdings wohl darauf zur\"uckzuf\"uhren ist, dass die mit der Messung abgedeckten Bereiche zu klein sind.

Zuletzt wird die Abh\"angigkeit vom Entladungsstrom $I_\text{E}$ untersucht.
Es zeigt sich, dass $n_e$, $\alpha$, $\omega_e$ und $\omega_i$ mit steigendem $I_\text{E}$ steigen; die anderen Werte fallen.
Der h\"ohere Strom (bei h\"oherer Spannung) f\"uhrt zu mehr Ionisation, was Ionisationsgrad und Elektronendichte erh\"oht.
Dies wiederum verst\"arkt die elektrische Abschirmung und senkt die Debye-L\"ange, w\"ahrend die Plasmafrequenzen steigen.
Wieder sinkt die Elektronentemperatur aufgrund h\"aufigerer St\"o\ss e.
%\printbibliography
%\FloatBarrier
\bibliography{mybib}{}
\end{document}
