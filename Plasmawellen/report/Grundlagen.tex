\section{Grundlagel}

\subsection{Plasma}
Ein Plasma ist ein teilweise oder vollst\"andig ionisiertes Gas.
Dies bedeutet, dass die Gasatome in positive Ionen und freie Elektronen aufspalten.
Der Ionisationsgrad
\begin{equation}
\alpha
    =\frac{n_i}{n}
    \label{eq:a}
\end{equation}
ist definiert als das Verh\"altnis der Dichte der Ionen $n_i$ zur Dichte der Ionen und Neutralteilchen $n$.
Plasmen sind quasineutral, was bedeutet, dass sie gleich viel positive wie negative Ladung enthalten und nach au\ss en hin neutral sind.
Ein Plasma kann aufgrund der freien Ladungen elektrische Felder abschirmen indem sich die negativen Ladungen gegen die positiven verschieben.
Die Feldst\"arke klingt dabei auf der Debye-L\"ange
\begin{equation}
\lambda_\text{D}\
    =\sqrt{\frac{\varepsilon_0 T_e}{e^2n_e}}
    \label{eq:lD}
\end{equation}
exponentiell ab.
Hierbei ist $\varepsilon_0$ die Dielektrizit\"atskonstante, $T_e$ die Elektronentemperatur (als Energie), $e$ die Elementarladung und $n_e$ die Teilchenzahldichte der freien Elektronen.
Die Debye-L\"ange schrumpft mit steigender Teilchenzahldichte und sinkender Temperatur, da sich langsamere Elektronen leichter von den elektrischen Feldern ablenken lassen.
Die Elektronen und Ionen k\"onnen anfangen zu schwingen wenn eine Ladungsverschiebung im Plasma erzeugt wird.
Die Eigenfrequenz dieser Schwingung ist die Plasmafrequenz.
F\"ur Elektronen ist sie gegeben durch
\begin{equation}
\omega_{\text{p},e}
    =\sqrt{\frac{e^2n_e}{\varepsilon_0m_e}},
    \label{eq:wpe}
\end{equation}
wobei $m_e$ die Masse eines Elektrons ist.
Die Ionen Schwingen aufgrund ihrer h\"oheren Masse $m_i$ wesentlich langsamer mit
\begin{equation}
\omega_{\text{p},i}
    =\sqrt{\frac{e^2n_i}{\varepsilon_0m_i}}.
    \label{eq:wpi}
\end{equation}
Ein Plasma muss die folgenden drei Kriterien erf\"ullen.
Zuerst muss die Abmessung des Plasmas $L$ wesentlich gr\"o\ss er als die Debye-L\"ange sein, da sonst keine kollektiven Eigenschaften in Erscheinung treten k\"onnen.
Um eine statistische Behandlung zu rechtfertigen muss eine gro\ss e Anzahl Teilchen $N_\text{D}$ innerhalb innerhalb einer Kugel mit der Debye-L\"ange als Radius (Debye-Kugel) vorhanden sein.
Ein Plasma kann nicht schneller als mit der Plasmafrequenz auf \"au\ss ere Felder reagieren.
Darum muss es m\"oglich sein eine Schwingung auszuf\"uehren ohne dass die Teilchen vorher sto\ss en.
Diese Bedingungen werden \"uber die Ungleichungen
\begin{align}
\omega_{\text{p},e}\tau
&>1,
\label{eq:c1}\\
\lambda_\text{D}
&\ll L,
\label{eq:c2}\\
N_\text{D}
&\gg 1
\label{eq:c3}
\end{align}
beschrieben, wobei $\tau$ die Sto\ss zeit der Elektronen im Plasma ist.
\cite{stroth11a}

\subsection{Wellen im Plasma}




\subsection{Langmuir-Sonde}
Eine Langmuir Sonde besteht aus einer Elektrode, die in ein Plasma eingebracht wird.
Die Elektrode ist an der Spitze leitend, aber ansonsten isoliert.
Das Potential der Sonde ist negativ gegen\"uber dem des Plasmas, da die Elektronen im Plasma wesentlich mobiler sind als die Ionen und somit h\"aufuger mit der Sonde stossen.
Nun kann man zus\"atzlich eine Spannung an die Langmuir-Sonde anlegen.
Das andere Ende der Leitung m\"undet ins Plasma, was bedeuted, dass jetzt ein Strom durch die Langmuir-Sonde und das Plasma flie\ss t.
Durch die Aufnahme einer Kennlinie
\cite{anleitung1}

\section{Durchf\"uhrung}
\begin{figure}[htbp]
    \centering
    \begin{circuitikz}
        \coordinate (o) at (0,0);
        \coordinate (x) at (20,0);
        \coordinate (bx) at (4,0);
        \coordinate (by) at (0,4);
        \coordinate (r) at (8,0);
        \coordinate (y) at (8,4);
        \draw[white,
        fill=yellow,
        fill opacity=0.2]
        (0,0) rectangle ($(bx)+(by)$);
        \draw[white,
        fill=blue,
        fill opacity=0.2]
        (bx) rectangle ($2*(bx)+(by)$);
        %box
        \draw[]
        (0,0) rectangle ($2*(bx)+(by)$)
        ($0.5*(bx)+0.9*(by)$) node[] {source}
        ($1.5*(bx)+0.9*(by)$) node[] {target};
        %grid
        \draw[dashed]
        ($(bx)+(0,0.3)$) -- ($(bx)+(by)-(0,0.2)$)
        ($(bx)+(0.2,1.5)$) node[right, rotate=270] {Gitter};
        %anode
        \draw[]
        (0.3,-3) to[kinky cross=(o)--(x), kinky crosses=right]
        ($(0.3,0)+0.5*(by)$) --
        ($(0.5,0)+0.5*(by)$)
        ($(0.5,0)+0.2*(by)$) --
        ($(0.5,0)+0.8*(by)$)
        ($(0.7,0)+0.8*(by)$) node[right, rotate=270] {Anode};
        %filament1
        \draw[]
        (1.7,-1.3) to[kinky cross=(o)--(x), kinky crosses=right]
        (1.7,0.7) to[cute inductor={Filament 1}]
        (2.7,0.7) to[kinky cross=(o)--(x), kinky crosses=left]
        (2.7,-1.3)
        (1.7,-1.3) to[battery1]
        (2.7,-1.3) node[below] {$U_\text{H}$};
        %filament2
        \draw[]
        (5.7,-1.3) coordinate (f2a) to[kinky cross=(o)--(x), kinky crosses=right]
        (5.7,0.7) coordinate (f2b) to[cute inductor={Filament 2}]
        (6.7,0.7) coordinate (f2c) to[kinky cross=(o)--(x), kinky crosses=left]
        (6.7,-1.3) coordinate (f2d)
        (5.7,-1.3) to[battery1]
        (6.7,-1.3) node[below] {$U_\text{H}$};
        %bottomline
        \draw[]
        (0.3,-3) -- (10,-3)
        (1,-3) -- (1,0)
        (1.7,-1.3) --
        (1.7,-2) to [battery1, l=$U_\text{D}$, n=res]
        (1.7,-3) node[ground] {}
        (5.7,-1.3) --
        (5.7,-2) to [battery1, l=$U_\text{D}$, n=res]
        (5.7,-3)
        (4,0.3) to[kinky cross=(o)--(x), kinky crosses=left]
        (4,-0.5) to[kinky cross=(f2a)--(f2b), kinky crosses=right]
        (6,-0.5) to[kinky cross=(f2c)--(f2d), kinky crosses=right]
        (9.5,-0.5)
        (4,-0.5) to[C]
        (4,-1.5)
        (4.5,-2) node[oscillator] {};
        %topline
        \draw[]
        (1.25,4) -- (1.25,4.5)
        (0,4.5) rectangle node{Vakuum Pumpe} (2.5,5.5)
        (3.75,4) -- (3.75,4.5)
        (3,4.5) rectangle node{Barometer} (4.5,5.5)
        (6.5,4) -- (6.5,4.5)
        (5,4.5) rectangle node{Gaszufuhr (Ventil)} (8,5.5);
        %Gauges
        \draw[]
        (6,2.5) node[rground, rotate=270]{}
        to[kinky cross=(r)--(y), kinky crosses=right] (8.3,2.5)
        (8.3,2) rectangle node{Kurbel} (9.5,3)
        (5.65,2) node[]{$\longleftrightarrow$}
        (6.7,2.5) node[above]{Langmuir-}
        (6.7,2.5) node[below]{Sonde}
        (9.5,2.5) --
        (10,2.5) --
        (10.5,3) node[circ]{} --
        (10.5,3.5)
        (9.5,3.5) rectangle node{Verst\"arker} (11.5,4)
        (11.5,3.75)--(12,3.75)
        (12,3) rectangle (14,4.5)
        (13,4) node{Spektrum-}
        (13,3.5) node{Analysator};
        %close bottom
        \draw[]
        (9.5,-1) rectangle node{Oszilloskop} (11.5,0)
        (10.5,-1) -- (10.5,-1.5)
        (10,-2.5) rectangle node{PC} (11,-1.5)
        (10.5,0) -- (10.5,0.5)
        (10,0.5) rectangle node{R} (11,1.5)
        (10.5,1.5) -- (10.5,2) node[circ]{}
        (11,2.5) node{S}
        (10,-3) -|
        (12.5,1) to[battery1] (11,1)
        (12.1,1.4) node[]{$U_\text{L}$};
    \end{circuitikz}
    \caption{
        Double plasma device.
        Die Abbildung zeigt den Versuchsaufbau.
        Die Plasmakammer ist durch ein Gitter in zwei Kammern \glqq source\grqq\ und \glqq target\grqq\ geteilt.
        Durch eine Vakuumpumpe wird das Gas aus der Kammer abgesaugt, w\"ahrend neues Gas durch ein Ventil zugef\"uhrt wird.
        Dies erlaubt es einen konstanten Druck im Plasma einzustellen, der allerdings aufgrund eines defekten Barometers nicht gemessen werden konnte.
        Das Plasma wird durch die Filamente erzeugt.
        Durch sie wird ein starker Strom geschickt, der Elektronen ausl\"ost und ins Gas schleudert.
        Dort werden die Elektronen zur Anode hin beschleunigt und sto\ss en dabei mit den Gasatomen, was diese ionisiert.
        Das Gitter bekommt eine negative Ladung von etwa \SI{-30}{\volt}, was verhindert, dass Elektronen von einer Kammer in die andere wechseln.
        Es kann dar\"uber hinaus zus\"atzlich mit einer Wechselspannung betrieben werden um Wellen im Plasma anzuregen.
        Die Langmuir-Sonde ist an einem beweglichen Stab angebracht, dessen Position im Plasma durch eine Kurbel manuell eingestellt werden kann.
        Dies erm\"oglicht es Wellenl\"ange und Amplitude der Wellen zu bestimmen.
        \cite{anleitung2,taylor72a}
        }
    \label{fig:circ}
\end{figure}
Abbildung \vref{fig:circ} zeigt einen Schaltplan f\"ur den Versuchsaufbau.
Darin ist auch der \glqq double plasma device\grqq\ erkl\"art.



\cite{anleitung2}
\FloatBarrier
\cite{schwabedissen99a}
\cite{taylor70a}