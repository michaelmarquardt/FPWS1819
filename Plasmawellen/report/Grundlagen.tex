\section{Grundlagel}
\subsection{Plasma}
Ein Plasma ist ein teilweise oder vollst\"andig ionisiertes Gas.
Der Ionisationsgrad
\begin{equation}
\alpha
    =\frac{n_i}{n}
    \label{eq:a}
\end{equation}
ist definiert als das Verh\"altnis der Dichte der Ionen $n_i$ zur Dichte der Ionen und Neutralteilchen $n$.
Plasmen sind quasineutral, was bedeutet, dass sie gleich viel positive wie negative Ladung enthalten und nach au\ss en hin neutral sind.
Ein Plasma kann aufgrund der freien Ladungen elektrische Felder abschirmen indem sich die negativen Ladungen gegen die positiven verschieben.
Die Feldst\"arke klingt dabei auf der Debye-L\"ange
\begin{equation}
\lambda_\text{D}\
    =\sqrt{\frac{\varepsilon_0 T_e}{e^2n_e}}
    \label{eq:lD}
\end{equation}
exponentiell ab.
Hierbei ist $\varepsilon_0$ die Dielektrizit\"atskonstante, $T_e$ die Elektronentemperatur (als Energie), $e$ die Elementarladung und $n_e$ die Teilchenzahldichte der freien Elektronen.
Die Debye-L\"ange schrumpft mit steigender Teilchenzahldichte und sinkender Temperatur, da sich langsamere Elektronen leichter von den elektrischen Feldern ablenken lassen.
Die Elektronen und Ionen k\"onnen anfangen zu schwingen wenn eine Ladungsverschiebung im Plasma erzeugt wird.
Die Eigenfrequenz dieser Schwingung ist die Plasmafrequenz.
F\"ur Elektronen ist sie gegeben durch
\begin{equation}
\omega_{\text{p},e}
    =\sqrt{\frac{e^2n_e}{\varepsilon_0m_e}},
    \label{eq:wpe}
\end{equation}
wobei $m_e$ die Masse eines Elektrons ist.
Die Ionen Schwingen aufgrund ihrer h\"oheren Masse $m_i$ wesentlich langsamer mit
\begin{equation}
\omega_{\text{p},i}
    =\sqrt{\frac{e^2n_i}{\varepsilon_0m_i}}.
    \label{eq:wpi}
\end{equation}
Ein Plasma muss die folgenden drei Kriterien erf\"ullen.
Zuerst muss die Abmessung des Plasmas $L$ wesentlich gr\"o\ss er als die Debye-L\"ange sein, da sonst keine kollektiven Eigenschaften in Erscheinung treten k\"onnen.
Um eine statistische Behandlung zu rechtfertigen muss eine gro\ss e Anzahl Teilchen $N_\text{D}$ innerhalb innerhalb einer Kugel mit der Debye-L\"ange als Radius (Debye-Kugel) vorhanden sein.
Ein Plasma kann nicht schneller als mit der Plasmafrequenz auf \"au\ss ere Felder reagieren.
Darum muss es m\"oglich sein eine Schwingung auszuf\"uehren ohne dass die Teilchen vorher Sto\ss en.
Diese Bedingungen werden \"uber die Ungleichungen
\begin{align}
\omega_{\text{p},e}\tau
&>1,
\label{eq:c1}\\
\lambda_\text{D}
&\gg L,
\label{eq:c2}\\
N_\text{D}
&\gg 1
\label{eq:c3}
\end{align}
beschrieben, wobei $\tau$ die Sto\ss zeit der Elektronen im Plasma ist.
\cite{stroth11a}










\cite{stroth11a}




\cite{anleitung}

a\cite{schwabedissen99a}
\cite{taylor70a}