\section{Zusammenfassung}
In diesem Versuch wurde mit einem Doppel Plasma Apparat Wellen im Plasma untersucht. Begonnen wurde mit dem Entladungsstrom. Der Entladungsstrom in Abhängigkeit der Entladungsspannung steigt ab einem Wert sehr stark an und ändert sich danach kaum noch. Der Unterschied zwischen der Einzel- und der Doppel-Kammer ist, dass der Entladunsstrom bei dem Doppel-Kammer größer ist als bei der Einzel Kammer. Bei dem Zusammenhang zwischen Entladungsspannung und Heizstrom gibt es fast keinen Unterscheid zwischen der Einzel- und der Doppel-Kammer.  Der Entladungsstrom steigt mit dem Heizstrom, vermutlich exponentiell, an. Der Entladungsstrom in Abhängigkeit des Druckes ist bis zu einem Wert fast identisch für die Einzel- und Doppel-Kammer. Bei großen Drücken steigt bei der Einzel-Kammer der Entladungsstrom an, bei der Doppel-Kammer sinkt er aber. Beachtet werden muss, dass der Druck aufgrund des defekten Druckmessgerätes nur in der Skala des Ventils, das die Gaszufuhr zum Doppel Plasma Apparat regelt, angegeben werden kann. Da der Zusammenhang zwischen der Skala und dem Druck nicht bekannt ist, kann somit der Zusammenhang zwischen dem Druck und dem Entladungsstrom  von dem beschriebenen Verlauf abweichen. Mit der Plasma Oszillations Methode kann die Plasmadichte bestimmt werden. Die Plasmadichte steigt mit steigendem Druck, mit steigendem Heizstrom und mit steigender Entladungsspannung an, wobei mit steigendem Druck die Zunahmne am stärksten ist. Aus der Dispersionsrelation der Ionen Akustikwellen kann die Schallgeschwindigkeit
\begin{align}
  c_{\mathrm{s}}= 2741.35\ \mathrm{ms}^{-1}
  \end{align}
  bestimmt werden. Aus der Schallgeschwindigkeit kann die Elektronentemperatur
  \begin{align}
    T_{\mathrm{e}}=3.809 \cdot 10^4\ \mathrm{K}
    \end{align}
  berechnet werden. Der exponentielle Abfall der Amplitude kann durch die Dämpfungslängen angegeben werden. Die Dämpfungslänge hängt vom Druck ab und ist von der Größenordnung
  \begin{align}
    \lambda_{\mathrm{damp}} \approx 10\ \mathrm{cm}.
  \end{align}
  Der Übergang von lineraren Ionen Akustikwellen zu nicht linearen Schockwellen kann durch Erhöhung der Amplitude erreicht werden. Mit steigender Amplitude ähnelt das gemessene Signal immer weniger einer Sinus-Funktion. Die Plasmakriterien, dass der Doppel Plasma Apparat deutlich größer als die Debye-Länge und die Teichlenanzahl in der Debye-Kugel deutlich größer als eins ist, sind erfüllt. Das dritte Kriteriuem, dass die Stoßzeit viel größer als die Periodendauer einer Schwingung des Plasmas sein muss, konnte aufgrund des defekten Druckmessgerätes nicht überprüft werden. 
  
