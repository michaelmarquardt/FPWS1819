\section{Auswertung}
\subsection{Kalibrierkurve}
Vor der eigentlichen Messung der Hysteresekurven, soll die Kalibrierkurve für das Magnetfeld gemessen werden. Dazu wird mittels einer Hall-Sonde das Magnetfeld gemessen. Die gemessenen Daten sind in der Abbildung \ref{fig:Kalibrierung} dargestellt. Die gemessene Kalibrierkurve ist nicht linear. Die Ursache für die Nichtlinearität ist die Sättigung des Eisenkerns. 
\begin{figure}[H]
\centering
\includegraphics[scale=0.8]{../Messdaten/auswertung/Kalibrierung.pdf}
\caption{ Das Magnetfeld für unterschiedliche Spannungen an der Position der Probe innerhalb der Spule. Zu sehen ist ein nichtlinearer Zusammenhang der Kalibrierkurve. }
\label{fig:Kalibrierung}
\end{figure}


\subsection{Temperaturabhängigkeit der Hysteresekurven}
Für die Berechnung des Kerr-Winkels wird der Zusammenhang zwischen der Drehung der Mikrometerschraube mit der gemessenen Spannung benötigt. Die gemessenen Werte sind in der Tabelle \ref{tab:Mikrometerschraube} dargestellt. Aus der Stellung der Mikrometerschraube kann der Winkel berechnet werden, wobei gilt $ 1 \mathrm{Tick}\ \widehat{=}\ 0.024\ {}^\circ$. Gut erkennbar ist der erwartete lineare Zusammenhang zwischen Kerr-Winkel und Spannung. Ohne das ein Fit notwendig ist, kann aufgrund der Daten aus der Tabelle \ref{tab:Mikrometerschraube} eine Geradengleichung bestimmt werden. Es gilt für alle gemessenen Spannungen das Verhältnis
\begin{equation}
1 \mathrm{Tick}\ \widehat{=}\ 0.01\ \mathrm{V}.
\end{equation}
Unter Berücksichtigung des Umrechnungsfaktors zwischen der Stellung der Mikrometerschraube und Winkel $\phi$ lautet die Geradengleichung 
\begin{equation}
\phi = 2.4\ \frac{{ }^\circ}{\mathrm{V}} \cdot U,
\label{eq:Kerr}
\end{equation}
wobei $U$ die Spannung ist. 

\begin{table}[h]
    \centering
    \caption{
        Zusammenhang zwischen Stellung der Mikrometerschraube des Polarisationsfilters und  des Kerr-Winkels mit der Spannung.
        }
    \label{tab:Mikrometerschraube}
    \begin{tabular}{r|c|l}
    Mikrometerschraube [ticks] & Kerr-Winkel [${}^\circ$]& Spannung [V] \\\hline
    0  & 0 & 0
 \\
    10 & 0.24 & 0.1
 \\
    20 & 0.48 & 0.2
 \\
    30 & 0.72 & 0.3
 \\
    40 & 0.96 & 0.4
 \\
    50 & 1.2 & 0.5 \\   
    \end{tabular}
\end{table}
Die unbearbeitete Hysteresekurve in der Abbildung \ref{fig:hysterese_original} ist schräg und nicht gerade. 
Die Ursache dafür ist, dass in der Polymerschicht die sich auf der magnetooptischen Disk befindet es auch zu einer Polarisationsänderung kommt. 
Diese Polarisationsänderung gibt es zusätzlich zu der Polarisationsänderung durch den Kerr-Effekt. 
Dieser Einfluss der Polymerschicht wurde aus den Hysterekurven rausgerechnet. 
Die korrigierten Hysteresekurven für unterschiedliche Temperaturen  sind in den Abbildungen \ref{fig:hysterese_temp_1} und \ref{fig:hysterese_temp_2} dargestellt. 


\begin{figure}[H]
\centering
\includegraphics[scale=0.8]{../Messdaten/auswertung/hysterese_7_original.pdf}
\caption{ Unbearbeitete Hysteresekurve bei einer Temperatur von $31,4\ {}^\circ$C.}
\label{fig:hysterese_original}
\end{figure}


\begin{figure}[H]
\centering
    \subfig{../Messdaten/auswertung/hysterese_7.pdf}
    \subfig{../Messdaten/auswertung/hysterese_8.pdf}
    \subfig{../Messdaten/auswertung/hysterese_9.pdf}
    \subfig{../Messdaten/auswertung/hysterese_10.pdf}
    \subfig{../Messdaten/auswertung/hysterese_11.pdf}
    \subfig{../Messdaten/auswertung/hysterese_12.pdf}
    
    
    
\caption{ Hysteresekurven für unterschiedliche Temperaturen.  }
\label{fig:hysterese_temp_1}
\end{figure}

\begin{figure}[H]
\centering
\subfig{../Messdaten/auswertung/hysterese_13.pdf}
    \subfig{../Messdaten/auswertung/hysterese_14.pdf}
    \subfig{../Messdaten/auswertung/hysterese_15.pdf}
    \subfig{../Messdaten/auswertung/hysterese_16.pdf}
    \subfig{../Messdaten/auswertung/hysterese_17.pdf}
\caption{ Hysteresekurven für unterschiedliche Temperaturen. }
\label{fig:hysterese_temp_2}
\end{figure}

Aus den Hysteresekurven kann der Kerr-Winkel für die Sättigung der Magnetisierung berechnet werden.
Dazu wird aus den Hysteresekurven die Differenz zwischen minimaler und maximaler Spannung abgelesen.
Die Häfte dieser Differenz entspricht dem Kerr-Winkel.
Die Spannung kann mittels der Gleichung \eqref{eq:Kerr} in den Winkel umgerechnet werden.
Die berechneten Kerr-Winkel für unterschiedliche Temperaturen sind in der Abbildung \ref{fig:Kerr_Winkel} dargestellt.
Wie zu erkennen ist, nimmt der Kerr-Winkel f"ur steigende Temperaturen leicht ab, wie es aus Abbildung \vref{fig:temperatur} erwartet wird.
\begin{figure}[H]
\centering
\includegraphics[scale=0.8]{../Messdaten/auswertung/Kerr_Winkel.pdf}
\caption{Der Kerr-Winkel für die Sättigung der Magnetisierung für unterschiedliche Temperaturen ermimttelt aus den Hysteresekurven.}
\label{fig:Kerr_Winkel}
\end{figure}


\begin{figure}[H]
\centering
\includegraphics[scale=0.8]{../Messdaten/auswertung/Kompensationstemperatur.pdf}
\caption{Die aus den Hysteresekurven ermittelten Koerzitivfeldstärken für unterschiedliche Temperaturen 
und der Fit mit der Gleichung \eqref{eq:koerzitivfeldstaerke}.}
\label{fig:koerzitivfeldstaerke}
\end{figure}

Die Koerzitivfeldstärke $H_\mathrm{c}$, also die magnetische Feldstärke die für die vollständige Entmagnetisierung notwendig ist, kann den Hysterekurven entnommen werden. 
Die Koerzitivfeldstärke entspricht der Hälfte der Breite der Hysterekurven. 
Die Koerzitivfeldstärken in Abhängigkeit der Temperatur sind in der Abbildung \ref{fig:koerzitivfeldstaerke} dargestellt. 
Aus den Koerzitivfeldstärken kann die Kompensationstemperatur $T_\mathrm{comp}$ bestimmt  
werden.
Dazu muss die Funktion  
\begin{equation}
H_\mathrm{c} = \frac{H_0 \cdot T}{T-T_\mathrm{comp}}
\label{eq:koerzitivfeldstaerke} 
\end{equation}
als Fitfunktion benutzt werden.
Die Fitparameter sind
\begin{align}
H_0 & = 59.70 \pm 4.78\ \mathrm{mT}\\
T_\mathrm{comp} & = 279.45 \pm 2.73\ \mathrm{K} .
\label{eq:fitwerte}
\end{align}

 
