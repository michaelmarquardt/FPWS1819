\subsection{Intensit"atsabh"angigkeit der Photolumineszenz}
\begin{figure}[htbp]
    \centering
    \input{../plots/umrechnung.pgf}
    \caption{
        Die Abbildung zeigt die Umrechnung des Winkels am Polarisator $\alpha$ in die durchgehende Laserleistung $P$.
        Die blaun Werte stehen dabei f"ur die als bekannt angegebenen Werte, w"ahrend die orangenen Werte als Mittelwert aus den beiden benachbarten Werten gebildet werden.
        Die Abbildung zeigt, dass im fraglichen Bereich eine etwa lineare Steigung vorherrscht, weswegen dies eine gute N"ahreung ist.
        Die vorhandenen Punkte werden alle f"ur die Messung in Abh"angigkeit der Intensit"at verwendet.
    }
    \label{fig:umrechnung}
\end{figure}
\begin{sidewaysfigure}
    \input{../plots/InAs_power.pgf}
    \caption{
        Spektrum der InAs QDs f"ur verschiedene Laserleistungen $P$.
        Es sind bis zu vier Peaks zu erkennen, die von rechts nach links der s, p, d und f-Schale im atom"ahnlichen Modell entsprechen.
        Die Kreuze bezeichnen die Messpunkte, w"ahrend die durchgezogenen Linien Gau\ss-Fits nach Gleichung \eqref{eq:fit} sind.
        F"ur $P=\SI{0}{\milli\watt}$ sind keine Peaks zu erkennen (da logischerweise keine Emission angeregt wird).
        Die H"ohe der Peaks steigt f"ur steigende Laserleistungen, ihre Position aber ver"andert sich nicht.
    }
    \label{fig:InAs_power}
\end{sidewaysfigure}
\begin{sidewaysfigure}
    \input{../plots/InP_power.pgf}
    \caption{
        Spektrum der InP QDs f"ur verschiedene Laserleistungen $P$.
        Der linke Peak entspricht der s-Schale.
        Die Kreuze bezeichnen die Messpunkte, w"ahrend die durchgezogenen Linien Gau\ss-Fits nach Gleichung \eqref{eq:fit} sind.
        Beim Fit wird nur der Bereich bis \SI{780}{\nano\metre} verwendet.
        %Wie zu erkennen ist, lassen sich die Peaks aber nicht besonders gut durch Gau\ss-Kurven ann"ahern.
        Die H"ohe der Peaks steigt f"ur steigende Laserleistungen, ihre Position aber ver"andert sich nicht.
    }
    \label{fig:InP_power}
\end{sidewaysfigure}
\begin{figure}[htbp]
    \centering
    \input{../plots/overpower.pgf}
    \caption{
        Die Abbildung zeigt die Intensit"aten der Peaks der s, p, d und f-Schale von InAs, sowie der s-Schale von InP.
        %Die gestrichelte Linie steht dabei f"ur die Ausgelesenen Maxima der Messdaten.
        %Sie ist mit aufgetragen, weil die Gau\ss-Kurven die s-Schale von InP nicht besonders gut approximieren.
        Die meisten Kurven zeigen eine sehr geringe Kr"ummung, also einen ann"ahrend linearen Verlauf.
        Dabei zeigt sich, dass bei der s-Schale die Steigung bei niedrigen Leistungen gr"o\ss er ist.
        Bei den anderen Schalen nimmt die Steigung bei st"arkeren Leistungen ein und nimmt dann zu.
        Die Intensit"at der s-Schale von InAs ist etwa dreimal so gro\ss\ wie die von InP.
    }
    \label{fig:overpower}
\end{figure}
Der versuch wird f"ur zwei verschiedene Halbleiter-Heterostrukturen durchgef"uhrt, InAs/GaAs (InAs) und InP/InGaP (InP).

Zuerst muss der Strahlengang justiert werden, indem der Laser auf den Spalt fokussiert wird.
Anschlie\ss end werden die Linsen eingebracht und die Photolumineszenz auf den Spalt fokussiert.
Zuletzt wird der reflektierte Laser mit einem Farbglasfilter, unmittelbar vor dem Spalt, geblockt und das Ergebnis durch weitere Feinjustage so eingestellt, dass die Peaks im gemessenen Spektrum maximiert werden.
Aufgrund einer verstellten Grundeinstellung sind f"ur InAs nur maximal vier Peaks zu beobachten.

Die Versuche wurden bei Temperaturen um die \SI{3.59}{\kelvin} bei InAs und \SI{3.85}{\kelvin} bei InP durchgef"uhrt.

Die Intensit"at des anregenden Lasers wird "uber einen Polarisator eingestellt.
Die verwendeten Leistungen $P$, und wie diese in den Polarisatorwinkel $\alpha$ umgerechnet werden, sind in Abbildung \vref{fig:umrechnung} aufgetragen.

Die gemessenen Spektren sind in den Abbildungen \vref{fig:InAs_power} und \vref{fig:InP_power} gezeigt.
An die Peaks werden Gauss-Funktionen der Form
\begin{equation}
f(\lambda)
    = a\cdot\exp(-\frac{(\lambda-\mu)^2}{2\sigma^2})
    \label{eq:fit}
\end{equation}
gefittet, wobei $a$, $\mu$ und $\sigma$ als Fitparameter verwendet werden.
Die resultierenden Fitparameter sind in den Tabellen \vref{tab:fitBP} und \vref{tab:fitAP} aufgelistet.

F"ur InAs sind bis zu vier Peaks zu erkennen, die von hohen zu niedrigen Wellenl"angen mit s, p, d und f bezeichnet werden, analog zu den Atomschalen im Schalenmodell.
F"ur niedrige Eingangsleistungen verschwindet der f-Peak.
InP zeigt einen s-Peak bei etwa \SI{750}{nm} und einen Haufen "uberlappender Peaks nahe \SI{810}{nm}.
F"ur beide Proben steigt die Peakh"ohe mit steigender Laserleistung, die Positionen der Peaks bleiben aber, wie erwartet, unver"andert.
Die Intensit"aten sind in Abbildung \vref{fig:overpower} "uber der Leistung dargestellt.