\subsection{Temperaturabh"angigkeit der Photolumineszenz}
\begin{sidewaysfigure}
    \input{../plots/InAs_temperature.pgf}
    \caption{
        Spektrum der InAs QDs f"ur verschiedene Temperaturen $T$.
        Es sind bis zu vier Peaks zu erkennen, die von rechts nach links der s, p, d und f-Schale im atom"ahnlichen Modell entsprechen.
        Die Kreuze bezeichnen die Messpunkte, w"ahrend die durchgezogenen Linien Gau\ss-Fits nach Gleichung \eqref{eq:fit} sind.
        Wenn die Temperatur erh"oht wird sinkt nicht nur die H"ohe der Peaks sondern sie werden auch zu h"oheren Wellenl"angen (niedrigeren Energien) verschoben.
    }
    \label{fig:InAs_temp}
\end{sidewaysfigure}
\begin{sidewaysfigure}
    \input{../plots/InP_temperature.pgf}
    \caption{
        Spektrum der InP QDs f"ur verschiedene Temperaturen $T$.
        Der linke Peak entspricht der s-Schale.
        Die Kreuze bezeichnen die Messpunkte, w"ahrend die durchgezogenen Linien Gau\ss-Fits nach Gleichung \eqref{eq:fit} sind.
        Beim Fit wird nur der Bereich bis \SI{780}{\nano\metre} verwendet.
        F"ur steigende Temperaturen nehmen die Peaks ab und werden zu h"oheren Wellenl"angen (niedrigeren Energien) verschoben.
        Die rechten Peaks verschwimmen immer mehr.
    }
    \label{fig:InP_temp}
\end{sidewaysfigure}
\begin{figure}[htbp]
    \centering
    \input{../plots/overtemp.pgf}
    \caption{
        Die Abbildung zeigt die Intensit"aten der Peaks der s, p, d und f-Schale von InAs, sowie der s-Schale von InP.
        %Die gestrichelte Linie steht dabei f"ur die Ausgelesenen Maxima der Messdaten.
        %Sie ist mit aufgetragen, weil die Gau\ss-Kurven die s-Schale von InP nicht besonders gut approximieren.
        F"ur steigende Temperaturen nimmt die Intensit"at f"ur InAs ab \SI{50}{\kelvin} stark ab.
        Die Schalen verschwinden nach und nach.
        Bei \SI{170}{\kelvin} sind keine Schalen mehr zu erkennen.
        F"ur InP hingegen ist die Abnahme der Intensit"at wesentlich geringer, sogar mit umgekehrter Kr"ummung.
        Auch bei \SI{170}{\kelvin} ist noch der s-Peak zu erkennen.
    }
    \label{fig:overtemp}
\end{figure}
\begin{figure}[htbp]
    \centering
    \input{../plots/mu_overtemp.pgf}
    \caption{
        Die Abbildung zeigt die Zentralwellenl"angen der Peaks der s, p, d und f-Schale von InAs, sowie der s-Schale von InP.
        F"ur h"ohere Temperaturen verschiebt sich der Peak zu h"oheren Wellenl"angen.
        Zu beachten ist, dass die Werte f"ur die d und f-Schale bei hohen Temperaturen nicht Aussagekr"aftig sind, da die Peaks hier bereits nahezu unkenntlich sind.
        Wenn man dies Ber"ucksichtigt, so ist die Verschiebung f"ur alle Schalen etwa gleich.
        InAs und Inp zeigen in etwa das gleiche Verhalten.
    }
    \label{fig:mu_overtemp}
\end{figure}
F"ur diesen Versuchsteil wird die h"ochste Laserleistung (\SI{2.69}{\milli\watt}) eingestellt.
Dann wird die K"uhlung der Probe heruntergefahren und ab \SI{10}{\kelvin} in Schritten von \SI{20}{\kelvin} das aktuelle Spektrum aufgenommen.
Die Ergebnisse sind in den Abbildungen \vref{fig:InAs_temp} und \vref{fig:InP_temp} dargestellt.
An die Peaks werden wieder Gau\ss-Kurven \eqref{eq:fit} gefittet.
Die Fitparameter sind in den Tabellen \vref{tab:fitBT} und \vref{tab:fitAT} aufgetragen.
In den Abbildungen \vref{fig:overtemp} und \vref{fig:mu_overtemp} sind die Intensit"at, beziehungsweise die Zentralwellenl"ange der Peaks in Abh"angigkeit der Temperatur aufgetragen.
Es zeigt sich, dass nicht nur die Intensit"at f"ur steigende Temperaturen abnimmt, sondern auch die Wellenl"ange zunimmt.
Die Wellenl"ange ist invers proportional zur Frequenz und somit zur Energie der ausgesendeten Photonen.
Mit steigender Temperatur wird also auch die Energiel"ucke zwischen den Zust"anden kleiner, die f"ur die Emission verantwortlich sind.