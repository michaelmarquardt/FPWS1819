\subsection{Aufl"osungsverm"ogen des Spektrometers}
\begin{figure}[htbp]
    \centering
    \input{../plots/na.pgf}
    \caption{
        Spektrum der Natrium-Dampflampe f"ur verschiedene Spalt"offnungen (Legende).
        Alle aufgenommenen Spektren zwischen \SI{0}{\micro\metre} und \SI{500}{\micro\metre} sind optisch nicht voneinander zu unterscheiden und deshalb nicht aufgetragen.
        Bei \SI{550}{\micro\metre} scheint sich etwas zu tun, da das Spektrum verschoben ist; allerdings handelt es sich dabei wohl eher um eine Verschiebung als eine "Offnung des Spalts, da die Intensit"at nicht gr"o\ss er wird.
        Das Spektrum wurde allerdings erst nach \SI{600}{\micro\metre} aufgenommen, was seine verl"asslichkeit in Frage stellt.
        Bei einer "Offnung von \SI{600}{\micro\metre} "ubersteuert die CCD-Kamera.
    }
    \label{fig:na}
\end{figure}
Die Neon-Dampflampe wird nun durch eine Natrium-Dampflampe ersetzt.
Die "Offnung des Spalts vor der CCD-Kamera wird dabei von \SI{0}{\micro\metre} bis \SI{600}{\micro\metre} variiert.
Der Offset wird dabei $-20$ Pixel gestellt, um das erwartete Spektrum zu treffen.
Die Spektren sind in Abbildung \vref{fig:na} dargestellt.
Es ist zu erkennen, das sich bis zu einer "Offnung von \SI{500}{\micro\metre} nichts ver"andert.
Bei \SI{600}{\micro\metre} "ubersteuert die CCD-Kamera, was nahelegt, dass der Spalt nun offen ist.
Die Verschiebung der Position bei \SI{550}{\micro\metre} kann dadurch erkl"art werden, dass der Wert erst nach der Messung von \SI{600}{\micro\metre} aufgenommen wurde.
Der Spalt wurde also wieder geschlossen, statt weiter ge"offnet und befindet sich nun an einer leicht anderen Position.
Die geringere Intensit"at kann mit einer tempor"aren Beeintr"achtigung der CCD-Kamera durch die "Ubersteuerung erkl"art werde.