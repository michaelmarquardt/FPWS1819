\subsection{Aufl"osungsverm"ogen des Spektrometers}
\begin{figure}[htbp]
    \centering
    \input{../plots/na.pgf}
    \caption{
        Spektrum der Natrium-Dampflampe f"ur verschiedene Spalt"offnungen (Legende).
        Alle aufgenommenen Spektren zwischen \SI{0}{\micro\metre} und \SI{500}{\micro\metre} sind optisch nicht voneinander zu unterscheiden und deshalb nicht aufgetragen.
        Bei \SI{550}{\micro\metre} scheint sich etwas zu tun, da das Spektrum verschoben ist; allerdings handelt es sich dabei wohl eher um eine Verschiebung als eine "Offnung des Spalts, da die Intensit"at nicht gr"o\ss er wird.
        Das Spektrum wurde allerdings erst nach \SI{600}{\micro\metre} aufgenommen, was seine verl"asslichkeit in Frage stellt.
        Bei einer "Offnung von \SI{600}{\micro\metre} "ubersteuert die CCD-Kamera.
    }
    \label{fig:na}
\end{figure}
Die Neon-Dampflampe wird nun durch eine Natrium-Dampflampe ersetzt.
Die "Offnung des Spalts vor der CCD-Kamera wird dabei von \SI{0}{\micro\metre} bis \SI{600}{\micro\metre} variiert.
Der Offset wird dabei auf $-20$ Pixel gestellt, um das erwartete Spektrum zu treffen.
Die Spektren sind in Abbildung \vref{fig:na} dargestellt.
Sie zeigen das Na-D-Liniendublett bei \SI{588,997}{nm} und \SI{589,593}{nm}.
Es wird durch die Spin-Bahn-Kopplung beim "Ubergang eines Elektrons von der 3p in die 3s Schale hervorgerufen.
\cite{NaD}
%Bei Beugung an einem Spalt gilt das Rayleigh-Kriterium
%\begin{equation}
%b = l\frac{\lambda}{d}
%\end{equation}
%f"ur das Aufl"osungsverm"ogen, wobei $\lambda$ die Wellenl"ange, $d$ die Spaltbreite und $l$ der Abstand zwischen Detektor und Spalt ist.
%Dann ist $b$ die L"angendifferenz, die am Detektor noch aufgel"ost werden kann.
%Liegen die beiden Peaks nun n"aher zusammen als $b$, so koennen sie nicht mehr aufgel"ost werden.
%Da sie aber noch aufgel"ost werden k"onnen, kann das Aufl"osungsverm"ogen der Kamera abgesch"atzt werden durch
%\begin{align}
%b = l\frac{\lambda}{d_\text{max}},
%\end{align}
%wobei $d_\text{max}$ die maximale Spalt"offnung ist, bei der noch beide Peaks zu erkennen sind.
%Der Abstand $l$ wird auf \SI{2}{cm} abgesch"atzt.

In der Abbildung ist zu erkennen, dass sich bis zu einer "Offnung von \SI{500}{\micro\metre} nichts ver"andert.
Bei \SI{600}{\micro\metre} "ubersteuert die CCD-Kamera, was nahelegt, dass der Spalt nun offen ist.
Die Verschiebung der Position bei \SI{550}{\micro\metre} kann dadurch erkl"art werden, dass der Wert erst nach der Messung von \SI{600}{\micro\metre} aufgenommen wurde.
Der Spalt wurde also wieder geschlossen, statt weiter ge"offnet und befindet sich nun an einer leicht anderen Position.
Die geringere Intensit"at kann mit einer tempor"aren Beeintr"achtigung der CCD-Kamera durch die "Ubersteuerung erkl"art werde.
Die Messung legt nahe, dass der Spalt sich nicht unter \SI{550}{\micro\metre} schlie{\ss}en l"asst.

Die im Versuch bestimmten Wellenl"angen der beiden Peaks sind bei \SI{588.548}{nm} und \SI{589.146}{nm}.
Ihre Differenz ist $\SI{598}{\pico\metre}$ was sehr nahe an der Differenz der Literaturwerte $\SI{596}{\pico\metre}$ ist.
Die Abweichung der absoluten Werte kann durch mangelnde Eichung der CCD-Kamera erkl"art werden.
F"ur das Aufl"osungsverm"ogen des Spektrometers ergibt sich dann
\begin{align*}
R
    &=\frac{\lambda}{\Delta \lambda}
    =\frac{\SI{589}{nm}}{\SI{598}{\pico\metre}}
    =985.
\end{align*}
