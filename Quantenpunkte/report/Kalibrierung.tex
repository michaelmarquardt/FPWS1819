\subsection{Kalibrierung des Spektrometers}
\begin{figure}[htbp]
    \centering
    \input{../plots/neon.pgf}
    \caption{
        Spektrum der Neon-Dampflampe.
        Der Spalt ist auf \SI{0}{\milli\metre} eingestellt und die Aufnahmezeit auf \SI{0.1}{\second}.
        Die Intensit"at $I$ ist auf [\si{counts\per\second}] normiert.
        Das Spektrum ist um $-45$ Pixel verschoben, um das Spektrum an die Erwartungen anzupassen.
    }
    \label{fig:neon}
\end{figure}
\begin{figure}[htbp]
    \centering
    \input{../plots/deckenlampe.pgf}
    \caption{
        Spektrum der Deckenlampe.
        Aufgenommen mit den selben Einstellungen wie f"ur die Neon-Dampflampe.
        Die Dampflampe wird dabei aus- und das deckenlicht eingeschaltet.
        Es sind drei Hauptpeaks um \SI{710}{\nano\metre} zu erkennen.
    }
    \label{fig:deckenlampe}
\end{figure}
Abbildung \vref{fig:neon} zeigt das aufgenommene Spektrum der Neon-Dampflampe.
Der Offset der CCD-Kamera wird auf $-45$ Pixel gestellt um die Peaks im Spektrum auf die richtige Position zu bringen, insbesondere den Peak bei \SI{703}{\nano\metre}.
Au\ss erdem stimmen die Peaks bei \SI{693}{\nano\metre}, sowie bei \SI{725}{\nano\metre} und \SI{660}{\nano\metre} gut "uberein.
Zus"atzlich ist das Spektrum der Deckenbeleuchtung mit den selben Einstellungen in Abbildung \vref{fig:deckenlampe} dargestellt.