\section{Zusammenfassung}
Im Versuch wurde das Temperatur- und Pumplaser-Leistungsabh"angige Verhalten der Fluoreszenz von Quantenpunkten untersucht.
Die Spektren wurden mit einem Gitterspektrometer aufgenommen, welches aus einem Gitter, welches die unterschiedlichen Frequenzen aufspaltet, und einer CCD-Kamera besteht, die die Spektrallinien dann misst.
Zuerst wurde das Verhalten dieses Gitterspektrometers an Dampflampen untersucht um sich mit der Bedienung vertraut zu machen.
Dabei wurde zun"achst das Spektrum einer Neon-Dampflampe aufgenommen.
Die Position der CCD-Kamera relativ zum Gitter muss dabei "uber eine konstante Verschiebung (in Pixeln) kompensiert werden.
So wurde erreicht, dass die charakterisierenden Linien des Spektrums (zum Beispiel \SI{703}{nm}) alle bei der richtigen Wellenl"ange sind.

Anschlie\ss end wurde das Aufl"osungsverm"ogen des Spektrometers mit einer Natrium-Dampflampe untersucht.
Dabei wurde das Natrium-D-Liniendublett bei \SI{589}{nm} f"ur verschiedene "Offnungsweiten des Spalts vor der Kamera betrachtet.
Es ergab sich, dass es bis zu einer Spaltbreite von $\lambda=\SI{550}{\micro\metre}$ zu erkennen war und dass der Wellenl"angenunterschied der beiden Peaks $\Delta\lambda=\SI{598}{\pico\metre}$ ist.
Daraus wurde das spektrale Aufl"osungsverm"ogen des Detektors zu $R=\lambda/\Delta\lambda =985$ bestimmt

Der Versuch wurde an zwei Arten von Quantenpunkten durchgef"uhrt: InAs/GaAs (InAs) und InP/InGaP (InP).
Dabei wurden die entsprechenden Quantenpunkte bei unter \SI{4}{\kelvin} mit einem Pumplaser bestrahlt.
Die Leistung wurde mithilfe eines Polarisators zwischen 0 und \SI{2.69}{\milli\watt} variiert.
Anschlie\ss end wurden noch Spektren f"ur Temperaturen bis \SI{170}{\kelvin} aufgenommen um die Temperaturabh"angigkeit zu untersuchen.

F"ur InAs ergeben sich vier sichtbare Gaussf"ormige Peaks bei Wellenl"angen zwischen \SI{865}{nm} und \SI{925}{nm} mit Abst"anden von etwa \SI{20}{nm}.
Diese k"onnen mit dem Schalenmodell Schalen zugeordnet werden, analog zu Atomschalen, wobei die gr"o{\ss}te Wellenl"ange der s-Schale entspricht, die n"achste p, dann d und schlie{\ss}lich f.
Zus"atzlich gibt es noch einen Peak von "Uberg"angen im Substrat, der bei h"oheren Energien liegt und deshalb kaum besetzt ist.
Es zeigt sich, dass die Peaks f"ur kleinere Pumpleistungen kleiner werden und die Peaks bei kleineren Wellenl"angen sukzessive verschwinden.
Dies geschieht, weil Schalen mit einer h"oheren Energie erst dann besetzt werden, wenn die niedrigeren Schalen ges"attigt sind.
Entgegen der Erwartungen sind die Peaks f"ur energetisch h"ohere Schalen niedriger, obwohl diese h"oher entartet sind und deshalb mehr besetzt sein sollten.
Dies kann mit mangelnder Fokussierung des Lasers erkl"art werden, da der Grundaufbau des Versuchs verstellt war.

"Ahnliches ist auch bei InP zu beobachten, nur dass hier lediglich ein s-Schalen Peak bei \SI{752}{nm} auszumachen ist.
Dar"uber hinaus ist bei gr"o{\ss}eren Wellenl"angen auch noch ein gro{\ss}er Mehrfachpeak zu sehen, der "Uberg"angen im Substrat zugeordnet werden kann.
Da sie bei niedrigeren Energien liegen wird die s-Schale nicht ges"attigt und h"ohere Schalen werden nicht besetzt.

Wenn die Temperatur der Quantenpunkte erh"oht wird, werden die Peaks kleiner.
Diese Abnahme der Intensit"at kann dadurch erkl"art werden, dass mit steigender Temperatur die Anzahl der Phononen und somit auch die Anzahl der Stöße zwischen Phononen und Elektronen zunimmt. 
Als erstes verschwinden die Peaks von dem Zustand mit der höchsten Energie, da die Elektronen dort am wenigsten Energie brauchen um den Quantenpunkt zu verlassen. 
Danach verschwinden die anderen Peaks nacheinander mit absteigender Energie. 


Dar"uber hinaus werden die Peaks rotverschoben.
Dies ist durch eine temperaturbedingte Verkleinerung der Bandl"ucke in den Halbleiter Heterostrukturen zu erkl"aren.
Die Peaks wandern dabei von \SI{10}{\kelvin} bis \SI{170}{\kelvin} um etwa \SI{20}{\nano\metre}.
