\section{Zusammenfassung}
Im Versuch wurde das Temperatur- und Pumplaser-Leistungsabh"angige Verhalten der Fluoreszenz von Quantenpunkten untersucht.
Die Spektren wurden mit einem Gitterspektrometer aufgenommen, welches aus einem Gitter, welches die unterschiedlichen Frequenzen aufspaltet, und einer CCD-Kamera besteht, die die Spektrallinien dann misst.
Zuerst wurde das Verhalten dieses Gitterspektrometers an Dampflampen untersucht um sich mit der Bedienung vertraut zu machen.
Dabei wurde zun"achst das Spektrum einer Neon-Dampflampe aufgenommen.
Die Position der CCD-Kamera relativ zum Gitter muss dabei "uber eine konstante Verschiebung (in Pixeln) kompensiert werden.
So wurde erreicht, dass die charakterisierenden Linien des Spektrums (zum Beispiel \SI{703}{nm}) alle bei der richtigen Wellenl"ange sind.

Anschlie\ss end wurde das Aufl"osungsverm"ogen der CCD-Kamera mit einer Natrium-Dampflampe untersucht.
Dabei wurde das Natrium-D-Liniendublett bei \SI{589}{nm} f"ur verschiedene "Offnungsweiten des Spalts vor der Kamera betrachtet.
Es ergab sich, dass es bis zu einer Spaltbreite von \SI{550}{\micro\metre} zu erkennen war.
Daraus wurde die 
%TODO Richtig, wenn Auswertung korrigiert ist.

