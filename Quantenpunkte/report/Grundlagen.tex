\section{Grundlagen}
\subsection{Quantenpunkte}
Ein Quantenpunkt ist eine Materialstruktur die meistens aus Halbleitern besteht. Dabei wird die Beweglichkeit der Ladungsträger, Löcher und Elektronen, in allen drei Raumdimensionen eingeschränkt. Die Materialstruktur ist also nulldiumensional, daher auch der Name Quantenpunkt.  Durch die Beschränkung sind die Energiezustände nicht mehr kontinuierlich, sondern diskret. Die Eigenschaften sind deshalb ähnlich wie bei einzelnen Atomen. Analog zu der Bildung von Molekühlen aus mehreren Atomen, kann aus mehreren Quantenpunkten auch  Quantenpunktmoleküle hergestelt werden. Die Elektronen haben dann die Möglichkeit durch Tunnelprozesse von einem Quantenpunkt zum nächsten zu gelangen. Der Unterschied zu den Atomen ist, dass bei Quantenpunkten die elektronischen und optischen Eigenschaften selber festgelegt werden können. Dies ist möglich, da die Größe, die Form und die Anzahl der Elektronenbei der Herstellung der Quantenpunkte selber gewählt werden können. \\
Für die Herstellung von Quantenpunkten gibt es verschiedene Verfahren. Bei der Nasschemischen Methode liegen die Nanopartikel als Kolloide in einer Suspension vor. Bei der selbstorganisierten Molekularstrukturepitaxie entstehen Quantenpunkte an der Grenzfläche zwischen dünnen Schichten von unterschiedlichen Halbleitern. Die Ursache für die selbstorganisierten Quantenpunkte liegt bei den unterschiedlichen Gitterkonstanten von den unterschiedlichen Halbleitern. Daduch kommt es zu Verspannugen und es ist energetisch günstiger das sich die  kleine Erhebungen in der Quantenpunktschicht bilden. Ein weiteres Verfahren ist die Lithographie, bei der die Quantenpunkte mittels Elektronenstrahlen oder ähnlichem auf ein Substrat   \glqq geschrieben\grqq{} werden. Danach müssen die Quantenpunkte noch \glqq freigelegt\grqq{} werden. \\
Damit die Quantenpunkte auch quantenmechanische Eigenschaften aufweisen, muss die Größe der Quantenpunkte in dem Bereich der DeBroglie-Wellenlänge liegen.  Mit den Ausdrücken für die Energie $ E = \mathrm{k}_{\mathrm{B}} T$ und $ E = p^2/(2m_{\mathrm{e}}^\star)$ ergibt sich
\begin{equation}
\lambda = \frac{h}{p} = \frac{h}{\sqrt{2m_{\mathrm{e}}^\star \mathrm{k}_{\mathrm{B}} T}} \approx 7.6\ \mathrm{nm},
\end{equation}
 wobei für die Temperatur $T=300$ K angenommen wird. Dieser Wert ist nur eine Näherung, da die effektive Masse der Elektronen $m_{\mathrm{e}}^\star$ materialspezifisch ist. Ein Quantenpunkt kann aus $10^3$ bis $10^9$ Atomen bestehen. \\
 Durch das Bescheinen eines Quatenpunktes mit einem Laser, tritt Fluoreszenz auf. Fluoreszenz ist die spontane Emission von Licht durch einen Elektronenübergang. Das dabei emittierte Licht ist im Allgemeinen energieärmer als das davor absorbierte Licht. Das absorbierte Licht stasmmt von dem Laser. Anderen Verwendungsmöglichkeiten von  Quantenpunkten sind die Verwendung als Einzelphotonenquellen und Laser. 
\subsection{Modelle für Quantenpunkte}
Das Verhältnis von der Höhe (z-Richtung) zur lateralen Ausdehnung  (x und y Richtung) ist klein. Deshalb gibt es in dem Potential des Quantenpunkt nur einen Zustand. Die laterale Ausdehnung kann man als eine zweidiemsionales System betrachten. Die Wechselwirkung von einzelnen Quantenpunkten muss bei der Beschreibung nicht berücksichtigt werden, da der Abstabd deutlich größer ist als die laterale Ausdehnung der Quantenpunkte. 

Das einfachste Modell eines Quantenpunktes ist ein Potentialtopf mit unendlichen hohen Wänden. Ein verbessertes Modell ist die Verwendung von endlich hohen Wänden, die die unterschiede zwischen dem Quantenpunkt und der Umgebung beschreiben. Ein besseres Modell ist ein zweidimensionaler harmonischer Oszillator für die laterale Ausdehnung. Nach der Abseparierung der z-Richtung lautet der Hamiltonian
\begin{equation}  
H = \frac{p_\mathrm{x}^2}{2 m^\star} + \frac{p_\mathrm{y}^2}{2 m^\star} + \frac{1}{2} m^\star \omega^2 x^2 + \frac{1}{2} m^\star \omega^2 y^2. 
\end{equation} 
Die bekannten Energieeigenwerte davon sind 
\begin{equation}
E_{\mathrm{x,y}} = \hbar \omega (2n +|l| +1)
\end{equation}
wobei die Hauptquantenzahl die Werte $n = 0,1,2 ..$ und die Nebenquantenzahl die Werte $l = 0, \pm1, \pm2, ...$ annehmen kann. 
Die Eigenfunktionen bestehen aus den Hermite Polynomen. Analog zur Atomphyisk kann der Parameter 
\begin{equation}
S = 2n +|l| = 0,1,2, ...
\end{equation}
definiert werden. Dies entspricht den Oribitalen (s,p,d,f) in Atomen. Für die Entartung ergibt sich 
\begin{equation}
d = 2(S+1),
\end{equation}
wobei der Faktor 2 von der Spinentartung kommt. Es kann gezeigt werden, das die Quantenzahlen bei den Übergängen erhalten bleiben müssen.  Es gilt somit die Auswahlregel $\Delta S =0$. 

\subsection{Spektrum von Quantenpunkten}
In dem Spetrum sind nicht alle erlaubten Übergänge des zweidiemsionalen harmonischen Oszillator Modelles gleich strak vertreten. Die Höhe der Peaks hängt von der Entartung ab. Je größer die Entartung eines Zustandes ist, desto größer ist der Peak. Der kleinste Peak sollte deshalb von dem s-Orbital kommen. Da dies nicht der einzige auftredente Effekt ist, muss dies in einem Experiment nicht der Fall sein. Eine Ursache dafür kann die Temperatur sein. Bei höheren Temperaturen nimmt der Einfluss der Phononen zu und es sind deshalb weniger Peaks sichtbar. Die Intensität des Lasers, der für die Fluoreszenz benötigt wird, veränert ebenfalls das Spektrum. Bei höheren Laserintensitäten sind mehr Peaks sichtbar. 
\subsection{Experimentelles}
